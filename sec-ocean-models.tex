\begin{frame}{Global circulation models}
%----------------------------------------------------------------------
  \begin{columns}
    \column{0.4\textwidth} {
      \includegraphics[width=\linewidth]{img/NASA-VIIRS_3Feb2012}
    }
    \column{0.6\textwidth} {
      \begin{itemize}
      \item<1-> \textit{Mathematical} descriptions of general circulation of
        % general circulation of  planetary
        atmosphere and oceans
      \item<2-> Basic components for \textit{climate models}
        \begin{itemize}
        \item Where other equations (e.g. chemical or
          biological) may be coupled
        \end{itemize}
        % \begin{itemize}
        % \item {Weather forecasting, climate understanding, predicting climate evolution...}
        % \end{itemize}
      % \item<3-> Increase of computer power $\Rightarrow$ becoming \textit{more feasible}
       \item<3-> Obtained from simplifications of extremely \textit{complicated
         equations} coming...
         \begin{itemize}
         \item from % different
           conservation laws in physics
           \strip{(momentum, mass, thermodynamics, \\ density, ...)}
         \item on a rotating sphere (the Earth)
         \end{itemize}
      \end{itemize}
      }
  \end{columns}
\end{frame}

\subsection{Quasi-Hydrostatic equations}
%======================================================================

\begin{frame}{The case of (large scale) ocean}
%----------------------------------------------------------------------
  \begin{itemize}\itemsep0.5em
  \item<1> \structure{The ocean:} A slightly compressible fluid
    endowed with Coriolis and buoyancy forces and a set of
    conservation laws from
    Physics
  \item<2-> Simplification of physical laws:\vspace{-1em}
    \begin{columns}
      \column{0.05\linewidth}
      \column{0.41\linewidth}
      \begin{enumerate}\itemsep1.5em
      \item<3-> Boussinesq hypothesis
        \tikz[na] \coordinate(Lboussinesq);
        \hfill
        \tikz[na] \coordinate(Rboussinesq);
      \item<4-> Cartesian coordinates
        \tikz[na] \coordinate(LbetaPlane);
        \hfill
        \tikz[na] \coordinate(RbetaPlane);
      \item<5-> Vertical scaling
        \tikz[na] \coordinate(LverticalScaling);
        \hfill
        \tikz[na] \coordinate(RverticalScaling);
      \end{enumerate}
      \column{0.54\linewidth}
      \begin{overprint}
        \onslide<3>
        \begin{block}{Boussinesq Hypothesis}
          \begin{itemize}
          \item \textit{Density} does not depart from a \textit{mean reference value},
            $\rho_\star>0$.
          \item Then it can be considered constant, $\rho_\star$.
          \item
            Except where it is multiplied by gravity acceleration $g$ (\textit{buoyancy} term)
            % Hence density can be replaced by the constant
            % $\rho_\star$ except in
            % \begin{itemize}
            % \item
              % \textit{buoyancy} term
            % \item conservation of \textit{energy equation}
              % {\color{PHDgrayC} (and state equation)}
          \end{itemize}
        \end{block}
        \onslide<4>
        \begin{block}{Cartesian Coordinates}
          \begin{itemize}
          \item A \textit{local projection} of the Earth surface will be
            assumed
          \item \textit{No loss of generality}: spherical coordinates requires
            only the proper handling of some terms
          \end{itemize}
        \end{block}
        \onslide<5>
        \begin{block}{Vertical Scaling of Domain}
          \vspace{-0.5em}
          \begin{itemize}
          \item The \textit{aspect ratio}
            \begin{equation*}
              \framedmath{\displaystyle\varepsilon = \frac{\text{vertical
                    scales}}{\text{horizontal scales}}}~
                \begin{tabular}[t]{l} is \large\textbf{small} \\[-0.2ex]
                  \tiny\it ~ $10^{-3}$, $10^{-4}$\end{tabular}
            \end{equation*}
%            \pause
          \item Dominant terms can be identified in
            \textit{vertical momentum} equation
          \item Numerical point of view: the problem is
            \textit{rescaled}, obtaining...
            \begin{itemize}
            \item \alert{\textbf{Isotropic domain}}
            \item \alert{\textbf{Anisotropic equations}}
            \end{itemize}
          \end{itemize}
        \end{block}
      \end{overprint}
    \end{columns}
  \end{itemize}

  \begin{tikzpicture}[overlay]
%        \draw<2> [thin, red,opacity=.8, fill=red,fill opacity=0.3](stone) circle (4pt);
%        \path<3>[->,>=latex, PHDredA, shorten >=4pt, opacity=.6]
%        (Lboussinesq) edge [out=0, in=130] (Rboussinesq);
        \draw<3>[->, PHDblue, ultra thick, opacity=.8] (Lboussinesq) -- (Rboussinesq);
        \draw<4>[->, PHDblue, ultra thick, opacity=.8] (LbetaPlane) -- (RbetaPlane);
        \draw<5>[->, PHDblue, ultra thick, opacity=.8] (LverticalScaling) -- (RverticalScaling);
\end{tikzpicture}
\end{frame}


\begin{frame}{The adimensional domain}
%  \vspace*{0.5em}
 After geometrical scaling, we obtain:
   $$
   \domain = \bigl\{ (\xx,z)\in \Rset^{3} \ /\ \xx=(x,y)\in\surface,\
   -D(\xx)< z < 0 \bigr\},
   $$
   where $\surface\subset\Rset^{2}$ is the surface domain and
   $D:\overline\surface \to \Rset_{+}$ a depth function
   \begin{center}
     \includegraphics[width=0.4\textwidth, height=5\baselineskip]{img/domain}
   \end{center}
   \begin{itemize}
   \item \alert{Rigid lid hypothesis} (flat surface) is introduced in
     this kind of domains
   \item Notation:
     \begin{itemize}
     \item $\surfaceBoundary$: part of the boundary corresponding to
       the surface
     \item $\bottomBoundary$: part corresponding to the bottom
       boundary
     \item $\talusBoundary$: lateral walls (if any)
     \end{itemize}
   \end{itemize}
 \end{frame}


\begin{frame}{Whole set of Equations}
  %----------------------------------------------------------------------
%  \vspace{-0.2em}
  \note{Comment briefly the main difficulties of Navier-Stokes equations}
  \begin{overprint}
    \onslide<1>
    In the adimensional domain:
    \onslide<2>
    \alert{Aspect ratio}:\tikz[na] \coordinate(LaspectRatio);
    \  multiplying vertical velocity terms!
    \onslide<3>
    \alert{Coriolis} force
    \tikz[na] \coordinate(Lcoriolis);
    \onslide<4>
    \textbf{Coupling} through \alert{\textbf{density}}~
    \tikz[na] \coordinate(Lcoupling);
  \end{overprint}
  \begin{block}{\small Conservation of momentum and continuity}
    \vspace{-0.66\baselineskip}
    \begin{align*}
      \dt \uu + \uu\cdot\gradx\uu + \vv\,\dz\uu  - \Delta_\visc\uu
      + \frac 1 \rho_\star \gradx \pp=
       \tikz[na] \node (Rcoriolis) {\framedmath<3>{\fU}};
      \\
      \tikz[na] \coordinate(RaspectRatio);
        \framedmath<2>{ \varepsilon^2 \Big( \dt \vv + \uu\cdot\gradx\vv +
          \vv\,\dz\vv - \Delta_\visc\vv \Big) }
        \displaystyle
        + \frac 1 \rho_\star \dz \pp +
        \frac{
          \tikz[na] \coordinate(RcouplingA);
          \framedmath<4>{\rho}\,\gravity}{\rho_\star} =
        0\hskip+0.5em
      \\[-0.2em]
      \div\uu + \dz\vv = 0\hskip+0.5em
    \end{align*}
    \vspace{-1.4\baselineskip}
  \end{block}
  \begin{block}{\small Convection-diffusion of \textit{temperature} and
      \textit{salinity} + state equation (density)}
    \vspace{-0.66\baselineskip}
    \begin{align*}
      \dt \Te  + (\uu \cdot \gradx) \Te + (\vv\cdot\dz) \Te  - \nu_\Te\Delta \Te &= \fT
      \\[0.3em]
      \dt \Sa  + (\uu \cdot \gradx) \Sa + (\vv\cdot\dz) \Sa -
      \nu_\Sa\Delta \Sa &= \fS
      \\[0.3em]
      \tikz[na] \coordinate(RcouplingB);
      \framedmath<4>{\rho} = \rho_\star\big(1-\beta_\Te(\Te-\Te_\star) + \beta_\Sa(\Sa&-\Sa_\star)\big)
    \end{align*}
    \vspace{-1.4\baselineskip}
  % \item Equation of state (dependence of density in terms of
  %   temperature and salinity)
  %   where $\Te_\star$ and $\Sa_\star$ are given reference values.
  \end{block}
  \tikzstyle{RaspectRatio} = [draw, circle, minimum size=.5cm, node
  distance=1.75cm]
  \tikzstyle{redcircle} = [draw, circle, color=PHDredA, node distance=3cm,
  minimum height=2em]

  \begin{tikzpicture}[overlay]
    \path<2>[myarrow]
    (LaspectRatio) edge [out=320, in=130] (RaspectRatio);
    \path<3>[myarrow]
    (Lcoriolis) edge [out=0, in=130] (Rcoriolis);
    \path<4>[myarrow]
    (Lcoupling) edge [out=0, in=135] (RcouplingA);
    \path<4>[myarrow]
    (Lcoupling) edge [out=0, in=135] (RcouplingB);
        % \draw<3>[->, PHDblue, ultra thick, opacity=.8] (Lboussinesq) -- (Rboussinesq);
        % \draw<4>[->, PHDblue, ultra thick, opacity=.8] (LbetaPlane) -- (RbetaPlane);
        % \draw<5>[->, PHDblue, ultra thick, opacity=.8] (LverticalScaling) -- (RverticalScaling);
  \end{tikzpicture}
\end{frame}


\begin{frame}{The case of constant density}
  %----------------------------------------------------------------------
  In the first part of this work, we fix the following usual
  assumption:

  \begin{center}
    \alert{density is a known constant}, e.g. $\rho=\rho_\star>0$
  \end{center}
  \vspace*{0.66em}
  \begin{itemize}\itemsep0.6em
  \item<1-> Then fluid motion equations can be treated independently
  \item<2-> If \tikz[na] \node (LrhoG) {$\rho g$};
    is written as potential... then
    \uncover<3->{it can be absorbed by pressure
      \tikz[na] \coordinate(LpressureAbsorb);}

  \begin{overprint}
    \onslide<1>
    \begin{block}{\small Conservation of momentum and continuity}
      \vspace{-1.3\baselineskip}
      \begin{align*}
        \dt \uu + \uu\cdot\gradx\uu + \vv\,\dz\uu - \Delta_\visc\uu +
        \frac 1 \rho_\star \gradx \pp &= \fU
        \\
        \tikz[na] \coordinate(RaspectRatio); \varepsilon^2 \Big( \dt \vv
        + \uu\cdot\gradx\vv + \vv\,\dz\vv - \Delta_\visc\vv \Big)
        + \dz \pp
        + \rho\gravity
        &= 0
        \\
        \div\uu + \dz\vv &= 0
      \end{align*}
      \vspace{-1.4\baselineskip}
    \end{block}
    \onslide<2>
    \begin{block}{\small Conservation of momentum and continuity}
      \vspace{-1.3\baselineskip}
      \begin{align*}
        \dt \uu + \uu\cdot\gradx\uu + \vv\,\dz\uu - \Delta_\visc\uu +
        \frac 1 \rho_\star \gradx \pp= \fU
        \\
        \varepsilon^2 \Big( \dt \vv
        + \uu\cdot\gradx\vv + \vv\,\dz\vv - \Delta_\visc\vv \Big)
        + \dz \pp +
%        \tikz[na] \coordinate;
        \tikz[na] \node[draw, circle, color=PHDredA, fill
        opacity=1] (RrhoG) {$\rho\gravity$};
        = 0
        \\
        \div\uu + \dz\vv = 0
      \end{align*}
      \vspace{-1.4\baselineskip}
    \end{block}
    \onslide<3->
    \begin{block}{\small Conservation of momentum and continuity}
      \vspace{-1.3\baselineskip}
      \begin{align*}
        \dt \uu + \uu\cdot\gradx\uu + \vv\,\dz\uu - \Delta_\visc\uu +
        \frac 1 \rho_\star \gradx \pp= \fU
        \\
        \varepsilon^2 \Big( \dt \vv + \uu\cdot\gradx\vv + \vv\,\dz\vv -
        \Delta_\visc\vv \Big)
        +
        \tikz[na] \node[draw, circle, color=PHDredA, fill opacity=1]
        (RpressureAbsorb) {$\dz \pp$}; = 0
        \\
        \div\uu + \dz\vv = 0
      \end{align*}
      \vspace{-1.4\baselineskip}
    \end{block}
  \end{overprint}
  \vspace{0.66em}
\item<4-> This system will be called anisotropic equations...
  \end{itemize}

  \begin{tikzpicture}[overlay]
    \path<2>[->,>=latex, PHDredA, shorten >=4pt, opacity=.6, thick]
    (LrhoG) edge [out=-45, in=135] (RrhoG);
    \path<3>[->,>=latex, PHDredA, shorten >=4pt, opacity=.6, thick]
    (LpressureAbsorb) edge [out=-45, in=45] (RpressureAbsorb);
  \end{tikzpicture}
 \end{frame}

\begin{frame}{Anisotropic Navier-Stokes equations}
  %----------------------------------------------------------------------
  % \vspace*{0.5em}
  \begin{BlockNoTitle}
    \begin{tabular}{@{}l|>{$}r<{$}>{$}l<{$}@{}}
      \multirow{3}{*}{
        \begin{turn}{30}
          \small\aniNS
        \end{turn}
        }
        &
        \dt \uu + \uu\cdot\gradx\uu + \vv\,\dz\uu - \Delta_\visc\uu +
        \frac 1 \rho_\star \gradx \pp &= \fU
        \\[0.2em]&
        \varepsilon^2 \Big( \dt \vv + \uu\cdot\gradx\vv + \vv\,\dz\vv -
        \Delta_\visc\vv \Big)
        + \dz \pp &= 0
        \\[0.2em]&
        \div\uu + \dz\vv &= 0
      \vspace{-1.4\baselineskip}
    \end{tabular}
  \end{BlockNoTitle}
  \uncover<1-> {If $\varepsilon=1$, standard $O(1)$ Navier-Stokes
  \par  \hfill
  ...but $\varepsilon=1$ is  not realistic in geophysical domains (\exclamation)}
  \vfill
  \uncover<2->{
  Better approaches:}
  \begin{enumerate}\itemsep0.5em
  \item<2-> $\varepsilon=0$: Hydrostatic Navier-Stokes or Primitive Equations
  \item<3-> $0<\varepsilon\lesssim 10^{-3}$:
    ``\textit{\bfseries\alert{Quasi-hydrostatic}}'' Navier-Stokes Equations
    \vspace*{0.5em}
    \begin{itemize}
    \item<3-> More realistic case in geophysical domains, e.g.
      $$
      \varepsilon = \frac{\text{5,121 m}}{\text{3,700 km}}
      \quad
      \text{(Mediterranean Sea)}
      $$
    \end{itemize}
  \end{enumerate}
\end{frame}

\begin{frame}{Hydrostatic Navier-Stokes (or Primitive) equations}
%------------------------------------------------------------
  The simplification \framedmath{\varepsilon=0} is usually taken,
  obtaining:
\begin{block}{}
  \begin{tabular}{@{}l|>{$}r<{$}>{$}l<{$}@{}}
    \multirow{3}{*}{
      \begin{turn}{30}
        \small \hydNS
      \end{turn}
    }
    &
    \dt\uu + (\uu\cdot\gradx)\uu +\vv\dz \uu
    - \Delta_\visc\uu + \gradx\pp &= \ff \ \In\domain,
    \label{eq:HNS.a}
    \\
    &
    \dz\pp & = 0 \ \In\domain,
    \label{eq:HNS.b}
    \\
    &
    \divx\uu +  \dz\vv &= 0 \ \In\domain.
    \label{eq:HNS.c}
  \end{tabular}
\end{block}
\begin{itemize}
\item Originally derived from scale analysis~\cita{Pedlosky:1987,CushmanRoisin-Beckers:09}
\item Mathematically justified as limit of anisotropic Navier-Stokes \aniNS,
  when $\varepsilon\to 0$ in vertical momentum
  equation~\cita{Besson-Laydi:92,Azerad-Guillen:01}
% \item Complemented with adequate boundary
%   conditions.
% ~(\ref{eq:bc.1})--(\ref{eq:bc.3}). See that, now,
%   it is natural not imposing boundary conditions for $\vv$ on
%   $\talusBoundary$.
\end{itemize}
\end{frame}

% \begin{frame}{Vertical integration of Hydrostatic Navier-Stokes}
%   \begin{itemize}
%   \item Model widely studied (both theoretically and in numerical
%     simulations)
%   \item Almost all works use the following \emph{\structure{vertical integrated formulation
%           of the primitive equations}}: find $\uu:\domain\times (0,T)
%       \to \Rset^2$, horizontal velocity, and $\ps:\domain\times (0,T)
%       \to\Rset$, (artificial) surface pressure, such that
%       \begin{BlockNoTitle}
%         \begin{tabular}{@{}l|>{$}r<{$}>{$}l<{$}@{}}
%           \multirow{3}{*}{
%             \begin{turn}{30}
%               \small\reducedProblem
%             \end{turn}
%           }
%           &
%           \dt\uu + (\uu\cdot\gradx)\uu +\vv\dz \uu - \Delta_\visc\uu +
%           \gradx\ps = \ff & \In\domain,
%           \label{eq:EP.a}
%           \\
%           &
%           \divx\langle\uu\rangle = 0 & \In\domain,
%           \label{eq:EP.b}
%           \\
%           &
%           \uu = 0 \quad \On\bottomBoundary\cup\talusBoundary, \qquad
%           \visc_z\dz\uu = \gs &\On \surfaceBoundary,
%           \label{eq:EP.c}
%         \end{tabular}
%       \end{BlockNoTitle}
%       where $\vv$ and $\langle\uu\rangle$ are defined by:
%   $$
%   \vv(\xx,\zz,t) = \int_z^0 \divx\uu(\xx,s,t)\;ds, \qquad
%   \langle\uu\rangle(\xx,t) = \int_{-D(\xx)}^0 \uu(\xx,\zz,t) dz.
%   $$
%       \begin{itemize}
%     \item Integrate in $z$ the divergence equations in ~\hydNS
%     \item Use account the rigid lid hypothesis~(\ref{eq:bc.2})
%     \end{itemize}
%     we obtain the
%       next equivalent
% \end{itemize}
% \end{itemize}
% \end{frame}

\begin{frame}{Vertical integration of Hydrostatic Navier-Stokes}
  \begin{itemize}\itemsep0.77ex
  \item Widely studied \soften{(both theoretically and in numerical
    simulations)}
  \item Almost all works \soften{use the following equivalent}
    \textbf{\alert{vertical integrated formulation of the primitive
        equations}}:
    \begin{itemize}
    \item find $\uu:\domain\times (0,T) \to \Rset^2$ (horizontal
      velocity) and
    \item $\ps:\surface\times (0,T) \to\Rset$ (artificial surface
      pressure) such that
    \end{itemize}
    \begin{BlockNoTitle}
      \begin{tabular}{@{}l|>{$}r<{$}>{$}l<{$}@{}}
        \multirow{3}{*}{
          \begin{turn}{30}
            \small\reducedProblem
          \end{turn}
        }
        &
        \dt\uu + (\uu\cdot\gradx)\uu +\vv\dz \uu - \Delta_\visc\uu +
        \gradx\ps = \ff & \In\domain,
        \label{eq:EP.a}
        \\
        &
        \divx\langle\uu\rangle = 0 & \In\domain,
        \label{eq:EP.b}
        \\
        &
        \uu = 0 \quad \On\bottomBoundary\cup\talusBoundary, \qquad
        \visc_z\dz\uu = \gs &\On \surfaceBoundary,
        \label{eq:EP.c}
      \end{tabular}
    \end{BlockNoTitle}
    where $\vv$ and $\langle\uu\rangle$ are defined by:
    \vspace{-0.5em}
    $$
    \vv(\xx,\zz,t) = \int_z^0 \divx\uu(\xx,s,t)\;ds, \qquad
    \langle\uu\rangle(\xx,t) = \int_{-D(\xx)}^0 \uu(\xx,\zz,t) dz.
    $$
  \item It is obtained...
    \begin{itemize}
    \item Integrating in $z$ the divergence equations in ~\hydNS
    \item Using the \textbf{rigid lid hypothesis}
    \end{itemize}
  \end{itemize}
\end{frame}

\begin{frame}
\frametitle{Comparison of the models}
\vspace{0.5em}
\small
\begin{tabularx}{\textwidth}{@{}X@{\qquad}X@{}}
  {\bf Not integrated \aniNS}  & {\bf Integrated model \reducedProblem}
  \\ \toprule
  System of  4 PDE +   4 unknowns: \ \bad &
  System of 3 PDE + 3 unknowns: \ \good
  \\ %\midrule
  % 4 unknowns \bad \ :
      \hspace{17ex}
      \begin{tabular}[l]{l}
      $\uu$ (in $\domain$), \\  $\vv$ (in $\domain$),
      \\ $\pp$ (in $\domain$) \ \bad
    \end{tabular}
    % $\vv=\vv(\uu)$
    &
    % 3 unknowns\ \good\ :
    \hspace{17ex}
    \begin{tabular}[b]{l}
      $\uu$ (in $\domain$) \\ $\ps$ (in $\surface$) \ \good
    \end{tabular}
    \\ \midrule
    $\vv$ is not decoupled\ \bad
    &
    $\vv=\vv(\uu)$ is decoupled \ \good
  \\ \midrule
  Only differential problems \ \good  &  Differential
  $+$ z-integral pb. \ \bad
  \\ \midrule
  Unstructured or struct. meshes \ \good &
  Structured meshes\ \bad
  \\
  % \begin{center}\vspace{-2em}
  \hspace{-3.2ex}
    \pgfimage[width=3.5cm]{img/unstruct-mesh}
  % \end{center}
  &
  % \begin{center}\vspace{-2em}
  \hspace{-3.2ex}
    \pgfimage[width=3.5cm]{img/struct-mesh}
  % \end{center}
%  \\ \bottomrule
    \\ \hline
    More general models, $\varepsilon\ge 0$\ \good &
    Needs simplification $\varepsilon=0$\ \bad
  \\ \midrule
  {\scriptsize Flexibility (extensions like mesh adaptivity)}\good &
  {\scriptsize Less flexibility for include extensions} \bad
\end{tabularx}
\end{frame}

\begin{frame}{Our objectives}
\begin{enumerate}\itemsep0.66em
\item \textit{Analysis} of the pecularities (the stability) of
  Navier-Stokes in geophysical domains (where $\varepsilon$ is ``small'')
\item Use this analysis to \textit{propose some new FE
    combinations} which:% fulfill the following \textit{purposes}:
  \begin{enumerate}\itemsep0.33em
  \item Allow numerically solving \textit{more realistic models than pure hydrostatic
      ones}. Moreover, to achieve solving both Hydrostatic and not
    Hydrostatic Navier-Stokes equations in the same way.
  \item \textit{Avoid the introduction of integro-differential equations},
    exploiting the advantages of non-integral models (enounced above)
  \end{enumerate}
\item Develop \textit{new discrete schemes} which are
  stable, when $\varepsilon\to 0$, for usual LBB FE combinations for
  Navier-Stokes (including $\varepsilon=0$)
\item Extend these new schemes to \textit{more complex time-dependent
    problems} (including variable density) and define \textit{new
  time-splitting schemes}
\end{enumerate}
\end{frame}


%%% Local Variables:
%%% coding: utf-8
%%% TeX-master: "numerical-oceanography"
%%% mode: latexv
%%% ispell-local-dictionary: "english"
%%% End:
