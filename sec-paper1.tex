
\begin{sectionframe}
  \vfill
  \begin{abstract}
    \begin{enumerate}
    \item Analysis of \textbf{Hydrostatic Stokes} equations show that
      they are \alert{not stable for ``classical'' Stokes LBB-stable FE
      combinations}. The reason is:
      \begin{itemize}
      \item An additional inf-sup condition is required
      \item It imposes richer FE spaces for horizontal velocity than
        vertical one
        % \begin{enumerate}
        % \item Usual Stokes LBB inf-sup condition
        % \item An additional ``hydrostatic'' inf-sup condition
        % \end{enumerate}
      \end{itemize}
    \item Stability analysis of \alert{classical Stokes} equations when
      unequal FE spaces are used for each component of the  velocity
    \item We provide some \textbf{new elements} which are
      \alert{stable for Hydrostatic Stokes} Equations
    \end{enumerate}
  \end{abstract}
\end{sectionframe}

% \subsection{Stability of Hydrostatic Stokes}
% %----------------------------------------------------------------------

% \begin{sectionframe}
%   \vfill
%   \begin{abstract}
%     \begin{enumerate}
%     \item<alert@*> Analysis of \textbf{Hydrostatic Stokes} equations.
%       They are \alert{not stable for ``classical'' Stokes LBB-stable FE
%       combinations}. The reason is:
%       \begin{itemize}
%       \item An additional inf-sup condition is required
%       \item It imposes richer FE spaces for horizontal velocity than
%         vertical one
%         % \begin{enumerate}
%         % \item Usual Stokes LBB inf-sup condition
%         % \item An additional ``hydrostatic'' inf-sup condition
%         % \end{enumerate}
%       \end{itemize}
%       \pgfsetfillopacity{0.33}
%     \item {\textbf{Stability analysis of classical Stokes} equations when
%       unequal FE spaces are used for each component of the  velocity}
%     \item We provide some \textbf{new elements which are
%       stable} for the Hydrostatic Stokes Equations
%     \end{enumerate}
%   \end{abstract}
% \end{sectionframe}

\begin{frame}{Numerical Experiment}
  \textbf{Stable FE approximations for Stokes} equations in a domain
  $$\alert{\Omega_{\epsilon}} = (0,1)\times(0,\alert{\varepsilon}) \subset \Rset^2$$
  \vspace{-1.2em}
  \begin{itemize}
    \setlength{\itemsep}{0.5em}
  \item Standard \structure{lid-driven cavity test}:
    \begin{itemize}
      \itemsep0.3em
    \item $v=0$ on $\Gamma_S\cup \Gamma_B$
    \item $u=0$ on $\Gamma_l \cup \Gamma_B$, traction B.C. on surface.
    \end{itemize}
  \item What happens if \structure{$\varepsilon \to 0$}? \hfill Test:
    $\varepsilon=1,10^{-1},10^{-2},\dots,10^{-5}$).
  \end{itemize}
  \medskip
  \textbf{Equivalently}: solving in the \textbf{unit square}
  $$\alert{\Omega} = (0,1)\times(0,1) \subset \Rset^2$$
  \vspace{-1.5em}
  \begin{BlockNoTitle}
    \begin{tabular}{@{}l|>{$}r<{$}>{$}l<{$}@{}}
      \multirow{3}{*}{
        \begin{turn}{30}
          \small\aniStokes
        \end{turn}
        }
        &
        - \Delta u + \gradx \pp &= 0 \quad \text{ in } \Omega
        \\[0.2em]&
        -\varepsilon^2
        \Delta v + \dz \pp &= 0 \quad \text{ in } \Omega
        \\[0.2em]&
        \divx u + \dz\vv &= 0 \quad \text{ in } \Omega
      \vspace{-1.4\baselineskip}
    \end{tabular}
  \end{BlockNoTitle}
\end{frame}

\begin{frame}{Usual FE are not valid for \aniNS when $\varepsilon\to 0$}
  \vspace*{0.25em}
  \begin{itemize}\itemsep0.33em
  \item Example: \P2-\P1 cavity test when $\varepsilon\to 0$
    \hfill{\small $\varepsilon=1,...,10^{-5}$)}
  \item \myframed{\textbf{Instabilities} when $\varepsilon\lesssim
      10^{-3}$ \hspace{5.2em}..due to vertical velocity}
    \note{Instabilities are due to anisotropy in domain (or in equations)}
  \end{itemize}
  \vspace*{0.25em}
  \begin{columns}
    ~\hspace{2em}
    \column{0.6\linewidth}
    \PlayVideoWithLabel[0.9\linewidth][0.85\linewidth][Streamlines]{%
      video/eps-to-zero-P2-P1-streamlines-20s.avi}
    \column{0.4\linewidth}
    \centering\PlayVideoWithLabel[0.62\linewidth][0.62\linewidth][Horizontal
    \ $\uu$]{%
      video/eps-to-zero-P2-P1-u-20s.avi}
    \\[0.3em]
%    \par\vskip-0.62em
    \centering\PlayVideoWithLabel[0.62\linewidth][0.62\linewidth][Vertical
    \ $\vv$]{%
      video/eps-to-zero-P2-P1-v-20s.avi}
    \end{columns}
  % \begin{center}
  %   \PlayVideoWithLabel[0.5\linewidth][0.5\linewidth][Streamlines]
  %   {video/eps-to-zero-P2-P1-streamlines.avi}
  % \end{center}
  \begin{flushright}
  \end{flushright}
\end{frame}


\begin{frame}{Why does the above occurs?}
\framesubtitle{A first approach}

\begin{itemize}\itemsep0.66em
\item We fix \textbf{the most unfavorable case}: $\varepsilon=0$

\item Let us define the spaces (which reflect \textit{boundary conditions}):
  \begin{equation*}
    \UU=\Uspace,\quad \VV=\Vspace,\quad \PP=\Pspace.
    \label{eq:spaces}
  \end{equation*}
\item And consider the (linear steady) hydrostatic
  Stokes problem:%
  \note{See that it's the kind of problem that arises when
    performing time discretizations of \hydNS}%
  % (``adequate spaces'') such that: = \Uspace\times
  % \Vspace\times\Pspace$ s.t.
  \begin{align*}
    &\text{Find } (\uu,v,p) \in U\times V \times P,
    \\
%  \qquad\qquad
  &\text{\hydStokes}
  \left\{
    \begin{aligned}
      - \visc_\xx\Delta\uu + \gradx\pp &= \ff_\xx &
      \begin{overprint}
        \onslide<2> \llaveizq{
          \alert{$\uu$: only primal variable}\\
          \alert{$\pp, \vv$: Lagrange multipliers}}
        \onslide<3>$\leftarrow \times\ \alert{\bar \uu \in \Uspace}$
      \end{overprint}
      \\ \dz\pp & =0 &
      \begin{overprint}
        \onslide<2>$\leftarrow \alert{\text{2nd. restriction}}$
        \onslide<3>$\leftarrow \times\ \alert{\bar v \in \Vspace}$
      \end{overprint}
      \\[0.1em]
      \divx\uu + \dz\vv &= 0 &
      \begin{overprint}
        \onslide<2>$\leftarrow \alert{\text{1st. restriction}}$
        \onslide<3>$\leftarrow \times \ \alert{\bar p \in \Pspace}$
      \end{overprint}
      \\
      +\ &\text{b. c.}
    \end{aligned}
  \right.
\end{align*}
\end{itemize}
\end{frame}



\begin{frame}{Analysis of the continuous problem}
% Find $(\uu,v,p) \in U\times V \times P =
% \Uspace\times\Vspace\times\Pspace$,
Weak formulation:
\begin{equation*}
  \left\{
    \begin{aligned}
      \visc_\xx\big(\grad u,\grad\bu\big)
      -\big(\divx\bu, p\big) &= \big(\ff_\xx,\bu\big) && \forall
      \bu\in \UU=\Uspace,
      \\
      \big(\divx \uu + \dz v,\bp\big) & \alert<2>{= 0} &&
      \forall\bp\in \PP=\Pspace
      \\
      \big(\dz \bv, p\big) & =0 &&
      \forall\bv\in \VV=\Vspace
    \end{aligned}
  \right.
\end{equation*}
\pause
Two Inf-Sup conditions are suggested:
\begin{alignat*}{4}
  \label{eq:ISp}
  \tag*{\ensuremath{(IS)^\PP}\xspace}
  &\quad&
  \sup_{0\neq(\uu,\vv)\in \UU \times \VV}
  \frac{(\div(\uu,\vv),\pp)}{\|\grad \uu,\dz \vv\|}
  &\ge \ConstISp \|\pp\|
  &\quad&
  \forall \pp \in \PP,
  \\
  \label{eq:ISv}
  \tag*{\ensuremath{(IS)^\VV}\xspace}
  &\quad &
  \sup_{0\neq \pp \in \PP}
  \frac{(\dz \vv,\pp)}{\|\pp\|}
  &\ge \ConstISv \|\dz\vv\|
  &\quad&
  \forall \vv\in \VV
\end{alignat*}

\textit{Remark}:
  \begin{itemize}
  \item \ISp: similar to Stokes LBB
    inf-sup condition (weaker?)~\cite{Brezzi-Fortin:91}
  \item \ISv: a new condition, first introduced by
    \textbf{P.~Azerad}~\cite{Azerad:1994,Azerad:PhD:96}
  \end{itemize}
\bigskip~
% \begin{equation*}
%   \begin{aligned}
%     \ISp\ & \alert<2>{
%       \sup_{0\neq( \uu, v)\in U\times V}
%       \frac{\big(\divx \uu +\dz v, p\big)}{||\grad \uu||_0 + ||\dz v||_0}
%       \ge
%       \beta_1 ||p||_0 \quad \forall p \in P
%     }
%     \\
%     \uncover<3->{
%       \ISv\ & \alert<3>{
%         \sup_{0\neq p\in P}
%         \frac{(\dz v, p)}{||p||_0}
%         \ge
%         \beta_2
%         ||\dz v||_0 \quad \forall v \in V
%       }
%     }
%   \end{aligned}
% \end{equation*}
\end{frame}

\begin{frame}{Existence, uniqueness and energy estimates}

  \begin{lemma}
    \ISp and \ISv are satisfied
  \end{lemma}
  \textbf{Proof}:
  \begin{itemize}
  \item \ISp $\Leftarrow$ Stokes inf-sup.
  \item \ISv for $v\in \Vspace$, take $p=\dz v\in \Pspace$
  \end{itemize}


  % Thanks \ISp and \ISv, we can provide an existence result,
  % \textit{not based on the reduced formulation} of \hydStokes:

\vfill

\begin{theorem}
  \begin{itemize}
  \item There exists an unique weak solution $(\uu,v,p)\in
    \UU\times\VV\times\PP$ of \hydStokes.
  \item
    Moreover, the following continuous dependence estimates hold:
    \begin{equation*}
      \label{eq:stability.estimates}
      \begin{aligned}
        ||\grad\uu||_0 \le ||\ff||_{-1}, && ||\dz\vv||_0 \le
        \frac{1}{\beta_2}||\ff||_{-1}, && ||p||_0 \le \frac{2}
        {\beta_1}||\ff||_{-1}
      \end{aligned}
    \end{equation*}
  \end{itemize}
  \label{theorem:1}
\end{theorem}
\end{frame}

\SetEmptyBackground
\begin{frame}
  \begin{proof}[Proof (P. Azerad~\cite{Azerad:PhD:96})]
    \begin{itemize}
    \item \structure{\textit{Step 1}}.
      Consider
      $\XX=\UU\times\VV\times\PP$ and, for each
      $\cc=(\uu,\vv,\pp)\in \XX$ and $\bc = (\bu, \bv, \bp)\in
      \XX$, \textit{define the forms}:
      \begin{align}
        \AA(\cc,\bc) & = \product{\grad \uu, \grad\bu} -
        \product{p,\div(\bu, \bv)} - \product{\div(\uu,v),\bp},
        \label{eq:HSoperator}
        \\
        \LL(\cc) & = \langle f,\bu \rangle,
        \label{eq:HSlinearform}
      \end{align}
      \begin{flushleft}
        \color{PHDgray}{
        \small \quad  $\hookrightarrow$ \textit{Notation}:
        $(\cdot,\cdot) \equiv  L^2(\domain)$ product, \
        $\langle\cdot,\cdot\rangle \equiv \UU^{-1}\times\UU$
        }
      duality.
      \end{flushleft}

      Then the weak formulation
      \hydStokes\ can be written as:
      \begin{equation}
        \label{eq:HS.weak.formulation}
        \text{Find}\ \cc\in \XX \quad\text{such that}\quad
        \AA(\cc,\bc) = \LL(\bc),
        \quad \forall \bc \in \XX.
      \end{equation}

    \item \structure{\textit{Step 2}}.
      Then, \ISp + \ISv $\Rightarrow$ ``\textit{generalized coercivity}'' for
      $\AA$ $\Rightarrow$ existence, uniqueness and bounds.
    \end{itemize}
  \end{proof}
\end{frame}
\SetDefaultBackground


\SetEmptyBackground
\begin{frame}
  \begin{proof}[A new proof  (via the \emph{saddle point}
     framework)]
%  Saddle point theory
  \begin{itemize}
  \item Let $\WW=\UU\times\VV$ endowed with
    $\|\ww\|_\WW=\|\nabla\uu,\dz\vv\|$
  \item Hydrostatic Stokes Problem $\equiv$ find $\ww \in \WW$
    and $\pp\in\PP$ such that
    \begin{align}
      \label{eq:saddle.point.proof.1}
      a(\ww,\bw) + b(\bw, p) &= \langle (\ff,0)^t,\bw \rangle_{\WW',\WW}  \quad\forall \bw \in \WW,
      \\
      \label{eq:saddle.point.proof.2}
      b(\ww,\bp) &=0  \quad\forall \bp\in \PP,
    \end{align}
    where, for each $\ww=(\uu,v)$ and $\bw = (\bu,\bv) \in \WW$:
    \begin{equation*}
      \begin{aligned}
        &a:\WW \times \WW \to \Rset, && a(\ww,\bw ) = \product{\grad
          \uu, \grad\bu} \leftarrow\text{\alert{\bf\itshape Not
            coercive} on } \UU\times\VV\exclamation
        \\
        &b:\WW \times \PP \to \Rset, && b(\ww,p) = -\product{\divx\uu
          + \dz\vv, p}
      \end{aligned}
    \end{equation*}
    \item But for well-posedness
      of~(\ref{eq:saddle.point.proof.1})--(\ref{eq:saddle.point.proof.2})
      it is sufficient$^*$ (see~\cite{Brezzi-Fortin:91}):
      \vspace{0.4em}
      \begin{enumerate}\itemsep0.4em
      \item $\displaystyle
          \sup_{\ww\in \WW}
          \frac{b(\ww,p)}{||\ww||_\WW} \ge \beta_0 ||p||_{P/\ker B^t},
          \quad\forall\pp\in\PP$
          \hfill $\leftarrow$ \ISp \qquad~
        \item $a(\cdot,\cdot)$ is \alert{\textbf{\itshape coercive on} $\ker B$}
          \hfill
        $\leftarrow$ \ISv \qquad~
      \end{enumerate}
    \end{itemize}
    \scriptsize{$^*$\color{PHDgrayB}{where $B: \WW \to \PP',\ \product{B\ww,p} =
          b(\ww,p) =\product{\ww,B^t p}$}}%
\end{proof}
\end{frame}
\SetDefaultBackground

\begin{frame}{The discrete problem}
  \begin{itemize}
  \item Let $\Uh$, $\Vh$, $\Ph$
    be conforming $\Uspace$, $\Vspace$, $\Pspace$ \
    finite element spaces
    on a (possibly unstructured) mesh ${\cal T}_h$
  \item Discrete problem: find $(\uu,v,p) \in \Uh\times \Vh \times \Ph$
    such that
\begin{equation*}
  \hydStokesDiscr
  \left\{
    \begin{aligned}
      \visc_\xx\product{\grad \uu,\grad\bu}
      -\product{\divx\bu, p} &= \product{f_\xx,\bu} && \forall
      \bu\in U_h,
      \\
      \product{\dz \bv, p} & =0 &&
      \forall\bv\in V_h,
      \\
      \product{\divx \uu + \dz v,\bp} & = 0 &&
      \forall\bp\in P_h.
    \end{aligned}
  \right.
\end{equation*}

  \item Let us define the discrete inf-sup conditions:
\begin{alignat*}{4}
  \tag*{\ensuremath{(IS)^\PP_h}\xspace}
  \label{eq:ISph}
  &\quad&
  \sup_{0\neq(\uh,\vh)\in \Uh \times \Vh}
  \frac{(\div(\uh,\vh),\ph)}{\|\grad \uh,\dz \vh\|}
  &\ge \ConstISph \|\ph\|
  &\quad&
  \forall \ph \in \Ph,
  \\
  \tag*{\ensuremath{(IS)^\VV_h}\xspace}
  \label{eq:ISvh}
  &\quad &
  \sup_{0\neq \ph \in \Ph}
  \frac{(\ph,\dz \vh)}{\|\ph\|}
  &\ge \ConstISvh \|\dz\vh\|
  &\quad&
  \forall \vh\in \Vh,
\end{alignat*}
%    \begin{equation*}
%   \begin{aligned}
%     \ISph\ &
%       \sup_{0\neq( \uu, v)\in U_h\times V_h}
%       \frac{\product{\divx \uu +\dz v, p}}{||\grad \uu||_0 + ||\dz v||_0}
%       \ge
%       \gamma_1 ||p||_0 \quad \forall p \in P_h
%     \\
%       \ISvh\ &
%         \sup_{0\neq p\in P_h}
%         \frac{\product{\dz v, p}}{||p||_0}
%         \ge
%         \gamma_2
%         ||\dz v||_0 \quad \forall v \in V_h
%   \end{aligned}
% \end{equation*}
\end{itemize}
\end{frame}

\begin{frame}{Existence and uniqueness of the discrete problem}
  \begin{theorem}
    The following conditions are equivalent:
    \begin{enumerate}
    \item The spaces $\Uh$, $\Vh$, $\Ph$ verify \ISph and \ISvh.
    \item $\boldsymbol{\exists!} \ (\uu,v,p)\in
      U_h\times V_h \times P_h \ \text{solution of \hydStokesDiscr}$.
    \end{enumerate}
    \medskip
    Furthermore, in this case: \vspace{-0.8em}
  \begin{equation*}
    \begin{aligned}
      ||\grad \uu||_0 \le C ||f||_{-1}, &&
      ||\dz v||_0  \le \frac C {\gamma_2}||f||_{-1}, &&
      ||p||_0  \le \frac{2C} {\gamma_1}||f||_{-1}
    \end{aligned}
  \end{equation*}
  \end{theorem}
  \medskip
  {\gris \textbf{Proof}: Similar to the continuous case. \hfill $\square$}
  % \medskip
  % \vfill
  % \begin{block}{Hydrostatic stabilty}
  %   \begin{itemize}
  %   \item We say \FEthreeSpaces\Uh\Vh\Ph\xspace is \structure{\em
  %       hydrostatic stable} if both \ISph and \ISvh hold
  %   \item Hence \textbf{hydrostatic stability is more restrictive}
  %     than usual Stokes stability (where only ``$\ISph$''
  %     is necessary)
  %   \end{itemize}
  % \end{block}
\end{frame}

\begin{frame}{A pragmatic sufficient condition for \ISph}

  \begin{lemma}[\textbf{Sufficient} condition for \ISph]
  \label{lemma:sufficient.condition.stokes.IS2}
  Assume that there exists $\Vstokes\subset\Vh\cap H_0^1(\Omega)$ such
  that the combination $\FEthreeSpaces{\Uh}{\Vstokes}{\Ph}$ is
  \colorUnderLine[PHDyellow]{Stokes-stable}\xspace (or LBB-stable), that is
    \begin{equation*}
      \sup_{0\neq (\uh,\vh)\in\Uh\times\Vstokes}
      \frac{(\div(\uh,\vh),\ph)}{||\grad\uh||_0+||\grad\vh||_0}
      \ge C ||\ph||_0 \quad\forall\pp\in\Ph.
    \end{equation*}
    Then the combination \FEthreeSpaces\Uh\Vh\Ph satisfy \ISph.
  \end{lemma}

  \textbf{Proof}: It is straightforward \hfill $\square$

  \begin{remark}[An open question]
    Is this condition necessary?,
    \\[0.3em]
    I.e., is there any combination \FEthreeSpaces\Uh\Vh\Ph which
    verifies \ISph and does not come from a Stokes-stable
    combination?
  \end{remark}
\end{frame}

\begin{frame}{Necessary conditions for hydrostatic stability}
\begin{lemma}[\textbf{Necessary} conditions for \ISph and \ISvh]
  \label{lemma:neccesary.conditions.IS1.and.IS2}
  \begin{enumerate}
  \item If $\ISph$ holds then \colorUnderLine[PHDyellow]{$\dim\Ph \le \dim\Uh + \dim\Vh$}.
  \item If $\ISvh$ holds then \colorUnderLine[PHDyellow]{$\dim\Vh \le \dim\Ph$}.
  \end{enumerate}
\end{lemma}

\begin{overprint}
 \onslide<2>
 \small \textbf{Proof}: Based on the analysis of $\ker{B^t}$ and a
  dimensional study of the structure of the \textit{underlying
    finite-dimensional linear system}. \hfill $\square$
  \onslide<3>
  \begin{corollary}
    The following ``classical'' elements are
    \colorUnderLine{\alert{\textbf{not stable}} for~\hydStokes}:
    \begin{enumerate}\itemsep0.33em
    \item \myframed{Taylor-Hood \FEtwoSpacesP21}
    \item \myframed{Mini-element \FEtwoSpacesP{1,b}1}
    \end{enumerate}
  \end{corollary}
  % \begin{proof}
  \medskip
  \textbf{Proof}:\vspace{-2em}
    \begin{center}
      \small
      \begin{tabular}{cc@{ / }c@{ -- }cccc}
        \mbox{FE} & $\Uh$ & $\Vh$ & $\Ph$ & \ISph & \ISvh & Stability
        \\ \otoprule
        $\FEtwoSpacesP21$ & \P2 & \P2 & \P1 & \good &
        \bad & \bad
        \\ \midrule
        $\FEtwoSpacesP{1,b}1$ & \P{1,b} & \P{1,b} & \P1 & \good &
        \bad & \bad
        \\ \bottomrule
      \end{tabular}
    \end{center}
%  \end{proof}
  \onslide<4>
  \begin{corollary}
    The following ``classical'' elements are
    \colorUnderLine{\alert{\textbf{not stable}} for~\hydStokes}:
    \begin{enumerate}\itemsep0.33em
    \item \myframed{Taylor-Hood \FEtwoSpacesP21}
    \item \myframed{Mini-element \FEtwoSpacesP{1,b}1}
    \end{enumerate}
  \end{corollary}
  \begin{remark}
    \begin{itemize}
    \item These combinations are neither stable for \aniNS when
      $\varepsilon\to 0$
    \item This is the \colorUnderLine{reason of the instabilities in previous animation}
    \end{itemize}
  \end{remark}
 \onslide<5>
  \bigskip
  But...
  how to find \textbf{\itshape \alert{stable combinations of FE}} for \hydStokes?
  \bigskip
%  \begin{remark}
    \begin{itemize}\itemsep1em
    \item Interesting equilibrium:
      \begin{itemize}\itemsep0.66em
      \item \colorUnderLine[PHDyellow]{$\dim(\Uh) + \dim(\Vh)$
          \textbf{must be big}} \hfill \color{PHDgrayC} ( $\ge\
        \dim(\Ph)$)\qquad~\mbox{ }
      \item \colorUnderLine[PHDyellow]{$\dim(\Vh)$ \textbf{must be
            small}} \hfill \color{PHDgrayC}( $\le \
        \dim(\Ph)$)\qquad~\mbox{ }
      \end{itemize}
    \item Natural answer: take \textbf{unequal approximations} for $\Uh$ and
      $\Vh$: %, so that% $\FEthreeSpaces\Uh\Vh\Ph$ with
      \begin{itemize}\itemsep0.66em
      \item \myframed{$\dim\Uh$  ``big''}
      \item \myframed{$\dim\Vh$ ``small''}
      \end{itemize}
    \end{itemize}
%  \end{remark}
\end{overprint}
\end{frame}


% >> 2d --- redefinitions
\renewcommand{\UU}{\ensuremath{U}\xspace}%
\renewcommand{\Uh}{\ensuremath{U_h}\xspace}%
\renewcommand{\uu}{u}%
\renewcommand{\uh}{\uu_{h}}%
\renewcommand{\divx}{\partial_x}%
%\renewcommand{\divuv}{\partial_x\uu + \partial_y\vv}
\renewcommand{\gradx}{\partial_x}%
% <<

\begin{frame}{Example 1: \FEthreeSpacesP210 is hydrostatic stable}
  \begin{itemize}\itemsep0.33em
  \item Let us consider the 2D combination \FEthreeSpaces\Uh\Vh\Ph
    $\sim$ \FEthreeSpacesP210\medskip
    \begin{itemize}\itemsep0.3em
    \item It is  Stokes-stable {\em\color{PHDgrayC}(see~\cite{Stenberg:90})}...
      \par\hfill
      ...hence it verifies \ISph\ {\em\color{PHDgrayC} (due to former
          sufficient conditon)}
    \item It also verifies \ISvh\ {\em\color{PHDgrayC}(because
        $\dz\Vh\subset\Ph$)}
    \end{itemize}
  \item Therefore \FEthreeSpacesP210 is stable for \hydStokes
  \item Numerical tests confirm this fact...
  \end{itemize}
%  \bigskip
  \begin{center}
    \centering\begin{tabular}{cc}
      \pgfimage[width=0.33\textwidth]{img/210-v-eps-0-uns} &
      \pgfimage[width=0.33\textwidth]{img/210-p-eps-0-uns}
      \\
      \small Streamlines & \small Pressure
    \end{tabular}
    \\%[0.5em]
    \scriptsize\itshape\color{PHDgrayC} Cavity test ($\varepsilon=0$),
    unstructured mesh with $h\simeq 1/40$
  \end{center}
\end{frame}


\begin{frame}{\FEthreeSpacesP210: error orders for \hydStokes}
  \small
  \newcolumntype{R}{>{\raggedleft\arraybackslash}X}%
  \begin{tabularx}{\textwidth}{ XR }
    \parbox[b]{\linewidth}{\begin{itemize}
    \item $\OO(h^2)$ for $(\uu,\vv)$ in $\|\cdot\|_{L^2}$
    \item $\OO(h)$ for $\pp$ in $\|\cdot\|_{L^2}$
    \item Note that, although $\uu\sim\P2$, optimal order $\OO(h^3)$
      is not reached (due to the influence of $\vv\sim\P1$)
    \end{itemize}
  ~}
    &
    \pgfimage[width=0.43\textwidth]{img/210-errors-loglog}
  \end{tabularx}
  \vspace{-1em}
  \begin{center}
    \scriptsize
    \begin{tabular}{l@{\qquad}r@{\qquad}rrrrrr}
      \\ \toprule
      $h$ &
      & $2^{-3}$ & $2^{-4}$ & $2^{-5}$ & $2^{-6}$ & $2^{-7}$  & $2^{-8}$
      \\ \otoprule
      \multirow{2}*{$\uu$}
      & $\|u-u_h\|_{L^2}$ & 1.099  & 1.709  & 1.952  & 1.876  & 1.988  & 2.019
      \\ \cmidrule{2-8}
      & $\|u-u_h\|_{H^1_0}$ & 0.580  & 0.838  & 1.018  & 0.926  & 0.989  & 1.093
      \\ \midrule
      \multirow{2}*{$\vv$}
      & $\|v-v_h\|_{L^2}$ & 1.156  & 1.500  & 1.924  & 1.476  & 1.457  & 2.072
      \\ \cmidrule{2-8}
      & $\|\dz v-\dz v_h\|_{L_2}$ & 0.770  & 0.894  & 1.032  & 0.941  & 1.004  & 1.081
      \\ \midrule
      $\pp$
      & $\|p-p_h\|_{L^2}$ & 0.479  & 0.790  & 1.022  & 0.905  & 0.977  & 1.110
      \\ \bottomrule
    \end{tabular}
  \end{center}
  % {Velocity and pressure errors for \FEthreeSpacesP 210
  %   Hydrostatic Stokes}
\end{frame}

% >> 3d --- redefinitions
\renewcommand{\UU}{\ensuremath{\mathbf{U}}}%
\renewcommand{\Uh}{\ensuremath{\UU_h}}%
\renewcommand{\uu}{\mathbf{u}}%
\renewcommand{\uh}{\uu_{h}}%
\renewcommand{\VV}{\ensuremath{V}}%
\renewcommand{\Vh}{\ensuremath{V_h}}%
\renewcommand{\vv}{v}%
\renewcommand{\vh}{v_{h}}%
\renewcommand{\divx}[1]{\nabla_\xx\cdot{}#1}
\renewcommand{\gradx}[1]{\nabla_\xx #1}
% <<

% \begin{frame}{Example 2. The ``Hydrostatic Finite Element''}
%   \begin{itemize}
%   \item Introduced by P.~Azerad~\cite{Azerad:PhD:96}
%   \item Defined by tensorial products of finite elements:
%     \begin{align*}
%       \uh &\ \sim\ \left({\cal R}_{1,h/2}(x,y) \times
%         {\cal P}_{1,h/2}(z)\right)^2, \\
%       \vh &\ \sim\ {\cal R}_{1,h}(x,y) \times {\cal P}_{1,h}(z), \\
%       \ph &\ \sim\ {\cal R}_{1,h}(x,y) \times {\cal P}_{0,h}(z),
%     \end{align*}
%     where ${\cal R}_{k,h}$ designs $\P k$ or $\Q k$ elements in a mesh
%     with size $h$
%   \item It needs a vertical structured mesh (prisms or
%     hexahedrals)
%   \end{itemize}
% \end{frame}

\begin{frame}{More hydrostatic stable elements?}
  About stability of \hydStokes in usual structured
  meshes:\bigskip
  \begin{itemize}\itemsep1em
  \item \FEthreeSpacesP210 \ \good
  \item \FEthreeSpacesP{1,b}10 \ \alt<1-2>{\question}{\myframed<3>{\bad}}
  \item \myframed<4>{\FEthreeSpacesP{1,b}11} \ \question
  \item \myframed<4>{\FEthreeSpacesP211} \ \question
  \end{itemize}
  \bigskip
  \uncover<2->{
    For each combination, we need testing both
    \begin{itemize}
    \item \ISvh $\leftarrow$ \myframed<2>{\textbf{Stokes inf-sup}}
    \item \ISph
    \item \only<4>{\myframed<4>{Work with F. Guillén, published in ApNumMath~\cite{fguillen-rrgalvan-paper1-ApNumMath:2016}}}
    \end{itemize}
  }
\end{frame}

\newcommand{\nv}{n}
\renewcommand{\imgdir}{img/}



\begin{frame}{\myframedAlert{\FEthreeSpacesP{1,b}11} verifies Stokes discrete~(\ref{eq:stokes:ISh}) }
  \begin{enumerate}\itemsep0.66em
  \item Not easy task: less degrees of freedom than ``\textbf{Mini-element}'' \exclamation
  \item We have proved that \FEthreeSpacesP{1,b}11 is stable in most
    unstructured 2D meshes
  \item This theory has been extended to 3D
  \item The tool employed: \textit{\alert{macroelements}} technique
    \par~\hfill\small(R.~Stenberg~\cite{Stenberg:84,Stenberg:90})
  \end{enumerate}
\end{frame}


\newcommand{\macroCondition}{\emph{macro-condition}\xspace}
\newcommand{\macroelemA}[2]{
  \def\yCoordA{#1}
  \def\yCoordB{#2}
  \begin{tikzpicture}
    \def\nv{4}
    \def\ix{1.6}\def\iy{1}
    \def\vertices{
      \coordinate (q0) at (0,0) [label=120:\scriptsize$q_0$];
      \coordinate (q1) at (-\ix*0.25,\iy);
%      \coordinate  (q2) at (-\ix*0.5,\iy);
      \coordinate  (q2) at  (-\ix,-0.25*\iy);
      \coordinate  (q3) at  (0.75*\ix,-\iy);
      \coordinate  (q4) at (\ix*1,0.25*\iy);
    }
    \def\showPoints{
      \foreach \i in {0,...,\nv}
        \fill [black] (q\i) circle (3pt);
    }
    \def\showLines{
      % Draw borders
      \draw (q1) -- (q2) -- (q3) -- (q4) -- cycle;
      % Draw interior edges
      \foreach \i in {1,...,\nv} {
        \draw (q\i) -- (q0);
      }
    }
    \def\showTriangles{
      \node at ($ (q0)!.5!(q1) $) [label=0:\scriptsize${T_1}$] {};
      \node at ($ (q0)!.5!(q2) $) [label=90:\scriptsize${T_2}$] {};
      \node at ($ (q0)!.5!(q3) $) [label=180:\scriptsize${T_3}$] {};
      \node at ($ (q0)!.5!(q4) $) [label=-90:\scriptsize${T_4}$] {};
    }
    \vertices
    \showPoints
    \showLines
    \showTriangles
  \end{tikzpicture}
}

\begin{frame}{Macroelements}
  \small
  \begin{columns}
    \column{0.76\linewidth}
    %---------------------
    \begin{itemize}\itemsep0.1em
      % \item Let $\Uh\sim\Pb$, $\Vh\sim\P1$, $\Ph\sim\P1$ continuous
      %   finite element spaces.
    \item Let $\Th$ a triangulation of $\domain\subset\Rset^2$
    \item Let $\Mh$ a \textbf{macroelement} partition of $\Th$
      \begin{flushright}\scriptsize
        (i.e. a connected sets which is union of at least two
        elements of $\Th$)
      \end{flushright}
    \item We consider macroelements with \alert{one
        only interior vertex}, $q_0$
    \item \textbf{Stenberg Theorem}: Under ``local to global''
      hypothesis, \alert{\textbf{next condition is sufficient for
          inf-sup}}:
    \end{itemize}
    \column{0.3\linewidth}
    %---------------------
    \vtop{\null\hbox{\macroelemA{-0.3}{-0.5}}}
  \end{columns}
 \bigskip
  \begin{equation*}
    \begin{aligned}
      \text{The set $N_M=\left\{ p\in \PM \st \int_M
          \div(u,v)\;p=0, \ \forall (u,v)\in
          \UM\times\VM \right\}$}
      \\[0.66em]
      \text{only consists of the functions which are \textbf{constant} on
        $M$}.
    \end{aligned}
  \end{equation*}
  It is called \textbf{Macro-condition}
%   \medskip
%   \begin{definition}%[\textbf{\amarillo Regular macroelement} condition]
%     In this case, we say \myframed{$M$ is \alert{\textbf{regular}}}
%     for \FEthreeSpaces{\Uh}{\Vh}{\Ph} \quad (or
%     \FEthreeSpaces{\Uh}{\Vh}{\Ph} is \alert{\textbf{regular}})
% %    \vspace{-1em}
%   \end{definition}
\end{frame}


% \begin{frame}{A sufficient condition for discrete Stokes inf-sup}
% %  Let's define:
%   We will use the well known result:
%   \vfill
%   \begin{theorem}[Stenberg'90]
%     Suppose that there is a fixed set of equivalence classes $E_i$ ,
%     $i = 1, 2, ..., l$, of macroelements and a positive integer $L$
%     such that:
%     \begin{description}
%     \item[$(M_1)$] \myframed{Each $M\in E_i$, $i = 1, 2, ..., l$,
%         \textbf{is regular}}
%     \item[$(M_2)$] Each $M \in M_h$ belongs to one of the classes $E_i$ , $i = 1, 2, ..., l$.
%     \item[$(M_3)$] Each interior edge of $\Th$ is interior to at least one and not more than $L$
%     macroelements of $M_h$
%     \end{description}
%     Then Stokes \alert{\textbf{inf-sup condition}} holds.
%   \end{theorem}
% \end{frame}


% \begin{frame}{Macroelements with one only interior vertex}
%   \begin{itemize}\itemsep0.5em
%     \item<1-> We will assume the following \alert{slightly restrictive} hypothesis on \Th:
%       {\it
%       \begin{itemize}
%       \item Each $T\in\Th$ has at least one vertex interior to
%         $\domain$
%       \item $\exists N\in\Nset$ (independ. $h$) s.t. each vertex of
%         $\Th$ is at most in $N$ elements
%       \end{itemize}
%     }
%   \item<2-> Then we will denote by \framedmath<2>{\MhOneVertex} to the macroelement
%     partition defined as follows:
%     \par\vspace{-2em}
%     \begin{tabular*}{0.5\linewidth}{@{\quad}p{0.5\linewidth}p{0.4\linewidth}}
%       \it~\par\vspace{1.5em}  Each $\Macro\in \MhOneVertex$ is union of elements
%       sharing \textbf{one vertex},
%       \framedmath<2>{q_0}
%       \tikz[na] \coordinate(intVertex);
%       &
%       \vtop{\null\hbox{\macroelemA{-0.3}{-0.5}}}
%     \end{tabular*}
%     \vspace{-2em}
%     % \begin{quote}
%     %   Each $\Macro\in \MhOneVertex$ is the union of all the elements
%     %   of $\Th$ sharing one vertex which is interior to $\domain$
%     % \end{quote}
%   \end{itemize}
%   \uncover<3>{
%   \begin{lemma}
%     \label{lemma:macroelements.with.one.interior.vertex}
%     The macroelement partition \MhOneVertex verifies $(M_2)$ and $(M_3)$.
%   \end{lemma}
% %  \smallskip
%   \begin{corollary}
%     If \Th is composed by \textbf{\alert{regular
%       macroelements}} of $\MhOneVertex$
%     % (i.e, $M_1$ \macroCondition holds $\forall
%     % M\in\MhOneVertex$),
%     then
%     \par\hfill\textbf{\alert{inf-sup condition}} holds
%   \end{corollary}
%   }
%   \begin{tikzpicture}[overlay]
%     \path<2>[myarrow] (intVertex) edge [out=0, in=230] (q0);
%   \end{tikzpicture}
% \end{frame}


% \begin{frame}{Structured and unstructured macroelements}
%   We have introduced some definitions:
%   \begin{definition}
%   A macroelement $M$, can be \structure{split by an hyperplane} $\Pi$
%   if there exists two other macroelements $M_1$ and $M_2$ such that:
%   \begin{center}
%     \llaveizq{$M_1\cup M_2=M$  and  \\
%       $M_1\cap M_2 \subset \Pi$}
%     \end{center}
%   \end{definition}

%   \begin{definition}
%     \begin{itemize}
%     \item A macroelement $M$ will be said \structure{\xStructured} if
%       it can be split by an hyperplane $x=C$, for some $C\in\Rset$
%       (i.e. \emph{orthogonal to $OX$ axis})
%     \item In  other case, it will be said \structure{\xUnstructured}
%     \end{itemize}
%   \end{definition}
%   \begin{itemize}
%   \item[$\star$] Analogue definitions for \structure\yUnstructured and
%     \structure\zUnstructured
%   \item[$\star$] For instance...
%   \end{itemize}
% \end{frame}

\newcommand{\bubbleMacro}{ % Bubble macroelement
  \begin{tikzpicture}
    \def\nv{5}
    \def\ix{2}\def\iy{1.8}
    \def\vertices{
      \coordinate (q0) at (0,0);
      \coordinate (q1) at  (-1.25*\ix,-0.25*\iy);
      \coordinate (q2) at  (-0.25*\ix,-\iy);
      \coordinate (q3) at  (0.75*\ix,-\iy);
      \coordinate (q4) at (\ix*1,0.5*\iy);
      \coordinate (q5) at (-\ix*0.25,\iy);
    }
    \def\showPoints{
      \foreach \i in {0,...,\nv}
      \fill [black] (q\i) circle (3pt);
    }
    \def\showLines{
      % Draw borders
      \draw (q1) -- (q2) -- (q3) -- (q4) -- (q5) -- cycle;
      % Draw interior edges
      \foreach \i in {1,...,\nv} {
        \draw (q\i) -- (q0);
      }
      \foreach \i in {1,...,\nv} {
        \draw [fill, lightgray] (x\i) circle (1.25pt);
        \draw (x\i) circle (2.25pt);
      }
    }
    \def\showTriangles{
      \node at ($ (q1)!.5!(q2) $) [label=230:\small${T_1}$] {};
      \node at ($ (q2)!.5!(q3) $) [label=-90:\small${T_2}$] {};
      \node at ($ (q3)!.5!(q4) $) [label=-15:\small${T_3}$] {};
      \node at ($ (q4)!.5!(q5) $) [label=90:\small${T_4}$] {};
      \node at ($ (q5)!.5!(q1) $) [label=130:\small${T_5}$] {};
    }
    \def\xPoints {
      \coordinate (x1) at ($ (q1)!.5!(q2)!.333!(q0) $);
      \coordinate (x2) at ($ (q2)!.5!(q3)!.333!(q0)  $);
      \coordinate (x3) at ($ (q3)!.5!(q4)!.333!(q0)  $);
      \coordinate (x4) at ($ (q4)!.5!(q5)!.333!(q0)  $);
      \coordinate (x5) at ($ (q1)!.5!(q5)!.333!(q0)  $);
    }
    \vertices
    \xPoints
    \showPoints
    \showLines
    \showTriangles
  \end{tikzpicture}
}


\newcommand{\unstableBubbleMacro}{ % Bubble macroelement where
  % P1b/P1/P1 is unstable
  \begin{tikzpicture}
    \def\nv{5}
    \def\ix{2}\def\iy{1.8}
    \def\vertices{
      \coordinate (q0) at (0,0);
      \coordinate (q1) at  (-\ix,0*\iy);
      \coordinate (q2) at  (-0.25*\ix,-\iy);
      \coordinate (q3) at  (0.75*\ix,-\iy);
      \coordinate (q4) at (\ix*1,0*\iy);
      \coordinate (q5) at (\ix*0.1,\iy);
    }
    \def\showPoints{
      \foreach \i in {0,...,\nv}
      \fill [black] (q\i) circle (3pt);
    }
    \def\showLines{
      % Draw borders
      \draw (q1) -- (q2) -- (q3) -- (q4) -- (q5) -- cycle;
      % Draw interior edges
      \foreach \i in {1,...,\nv} {
        \draw (q\i) -- (q0);
      }
      \foreach \i in {1,...,\nv} {
        \draw [fill, lightgray] (x\i) circle (1.25pt);
        \draw (x\i) circle (2.25pt);
      }
    }
    \def\showTriangles{
      \node at ($ (q1)!.5!(q2) $) [label=230:\small${T_1}$] {};
      \node at ($ (q2)!.5!(q3) $) [label=-90:\small${T_2}$] {};
      \node at ($ (q3)!.5!(q4) $) [label=-15:\small${T_3}$] {};
      \node at ($ (q4)!.5!(q5) $) [label=90:\small${T_4}$] {};
      \node at ($ (q5)!.5!(q1) $) [label=130:\small${T_5}$] {};
    }
    \def\xPoints {
      \coordinate (x1) at ($ (q1)!.5!(q2)!.333!(q0) $);
      \coordinate (x2) at ($ (q2)!.5!(q3)!.333!(q0)  $);
      \coordinate (x3) at ($ (q3)!.5!(q4)!.333!(q0)  $);
      \coordinate (x4) at ($ (q4)!.5!(q5)!.333!(q0)  $);
      \coordinate (x5) at ($ (q1)!.5!(q5)!.333!(q0)  $);
    }
    \vertices
    \xPoints
    \showPoints
    \showLines
    \showTriangles
    \node at ($ (q4)!.4!(q5)!-1!(q0) $) {};
    \node at ($ (q4)!.333!(q3)!-0.7!(q0) $) {};
  \end{tikzpicture}
}

\begin{frame}{{\amarillo\xStructured} and {\amarillo\xUnstructured} macroelements}
  \begin{columns}
    \column{0.5\linewidth}
    \unstableBubbleMacro
    \\
    \begin{itemize}
    \item \myframed{\textbf\xUnstructured}
       (it cannot be \structure{split} by any line
      orthogonal to $OX$ axis)
    \item \myframed{\textbf\yStructured}
      \ (can be \structure{split} by a line
      orthogonal to $OY$)
    \end{itemize}
    % --------
    \pause
    \column{0.5\linewidth}
    \bubbleMacro
    \\
    \begin{itemize}
    \item \textbf\xUnstructured\ macroelement
    \item Also \textbf\yUnstructured
    \end{itemize}
    \vspace{1.5em}~
   \end{columns}
\end{frame}

\begin{frame}{Regular macroelements for \FEthreeSpacesP{1,b}11}
  \begin{itemize}\itemsep1em
  \item The following result \colorUnderLine{relates
      \textbf{Macro-condition} with \textbf{structure}}
\item In fact, it states that ~\medskip
  \quad

  \begin{BlockNoTitle}
    \begin{center} \it bubbles can be
      \textbf{\itshape removed from one (only) component of velocity} \\
      \rotatebox{90}{$\Leftrightarrow$}
      \\
      $M$ is \it \ \textbf{\itshape unstructured in corresponding
        direction}
    \end{center}
  \end{BlockNoTitle}

\item More specifically...
  \begin{theorem}
    Let $M$ be a macroelement in $\MhOneVertex$.
    Then
    \begin{center}
      \colorUnderLine{\FEthreeSpacesP{1,b}{1}{1} verifies
        \textbf{Macro-condition} in $M$} $\Leftrightarrow$ $M$ is
      \colorUnderLine{\textbf{\yUnstructured}}
    \end{center}
  \end{theorem}
  \item[$\star$] Analogue result for \FEthreeSpacesP{1}{1,b}{1}
% \item[$\star$] For instance...
\end{itemize}
\end{frame}

% \begin{frame}{{\amarillo Regular} and {\amarillo not regular} macroelements}
%   \small
%   \begin{columns}
%     \column{0.5\linewidth}
%     \unstableBubbleMacro
%     \\
%     \begin{itemize}
%     \item \xUnstructured $\Rightarrow$
%       \par~ \hfill \myframed<1>{\FEthreeSpacesP1{1,b}1 \textbf{Macro-cond.}}
%     \item \yStructured $\Rightarrow$
%       \par~ \hfill \myframed<1>{\FEthreeSpacesP{1,b}11  \textbf{No
%           Macro-cond.}}
%     \end{itemize}
%     \vfill~
%     % --------
%     \pause
%     \column{0.5\linewidth}
%     \bubbleMacro
%     \\
%     \begin{itemize}
%     \item \xUnstructured $\Rightarrow$
%       \par~ \hfill \myframed<2>{\FEthreeSpacesP{1,b}11
%         \textbf{Macro-cond}}
%     \item \yUnstructured $\Rightarrow$
%       % \FEthreeSpacesP1{1,b}1 is  \textbf{regular}
%       \par~ \hfill \myframed{\FEthreeSpacesP1{1,b}1
%         \textbf{Macro-cond}}
%     \end{itemize}
%     \vspace{0.9em}
%    \end{columns}
% \end{frame}

\begin{frame}{Sufficient condition for stability}
  \begin{corollary}
    If a mesh is composed of \textbf{\yUnstructured macroelements}, then
    \alert{inf-sup condition holds for \FEthreeSpacesP{1,b}11} elements
  \end{corollary}
  \begin{center}
     \pgfimage[width=0.9\linewidth,height=7\baselineskip]{\imgdir/mesh-xStr-yUnstr}
  \end{center}
  \begin{itemize}\itemsep0.9em
  \item \textbf{Proof}: Apply Stenberg's Theorem.
  \item Analogue result for \FEthreeSpacesP1{1,b}1
  \end{itemize}
\end{frame}

% \begin{frame}{\myframedAlert{\FEthreeSpacesP{1,b}11 (or Mini-mini-element)}: advantages}
%   \begin{enumerate}[<+->]\itemsep0.5em
%   \item In practice, one can hope \textbf{\alert{inf-sup condition is
%         verified in most meshes}}, specifically those build by
%     \textbf{usual mesh generators}. Our \textbf{numerical tests
%       confirm} this fact
%   \item It can be shown the following \alert{\textbf{optimal error
%         estimates}}:
%     \begin{equation*}
%       \|\ww - \ww_h\|_{H^1} + \| p - p_h\|_{L^2} \le C\, h\quad \hbox{and}\quad
%       \|\ww - \ww_h\|_{L^2} \le C\, h^2
%     \end{equation*}
%     They agree with our \textbf{numerical tests}
%   \item We have found a ``\myframed<3>{\alert{\textbf{\itshape
%           Mini-mini-element}}}'': element
%     \begin{itemize}
%     \item with less d.o.f. than the mini element,
%     \item with optimal order and
%     \item stable in most meshes
%     \end{itemize}

%   % \item Previous theorem is \alert{\textbf{not a necessary condition}}
%   %   and inf-sup condition may be verified even if some $M$ are
%   %   structured
%   \item In fact, \alert{\textbf{counterexamples where inf-sup
%         condition is not verified}} have been found just in meshes having
%     \textbf{both a strong $x$-structure and $y$-structure}
%   \end{enumerate}
% \end{frame}

\begin{frame}{\myframedAlert{Stokes \FEthreeSpacesP{1,b}11}
    lid-driven cavity test, 1}
  \begin{tabular}{@{}c@{}c@{}c@{}}
    \pgfimage[width=0.32\linewidth]{\imgdir/unstruct-mesh}
    &
    \pgfimage[width=0.31\linewidth]{\imgdir/b1-1-v-uns}
    &
    \pgfimage[width=0.31\linewidth]{\imgdir/b1-1-p-uns}
  \end{tabular}
  \textbf{Unstructured mesh}

  \begin{tabular}{@{}c@{}c@{}c@{}}
    \pgfimage[width=0.32\linewidth]{\imgdir/struct-mesh}
    &
    \pgfimage[width=0.31\linewidth]{\imgdir/b1-1-v-str}
    &
    \pgfimage[width=0.31\linewidth]{\imgdir/b1-1-p-str}
  \end{tabular}
  \textbf{Structured mesh} %\FEthreeSpacesP{1,b}{1,b}1
\end{frame}

% \begin{frame}{Mesh structured in just 1 direction}
%   Test in a $x$--structured but $y$--unstructured mesh
%   \begin{center}
    % \pgfimage[width=0.5\linewidth,height=5\baselineskip]{\imgdir/mesh-xStr-yUnstr}
    % \\
  %   \begin{tabular}{@{}c@{}c@{}c@{}}
  %     \pgfimage[width=0.32\linewidth]{\imgdir/bb-1_x-struc_y_unstruct}
  %     &
  %     \pgfimage[width=0.32\linewidth]{\imgdir/b1-1_x-struc_y_unstruct}
  %     &
  %     \pgfimage[width=0.32\linewidth]{\imgdir/1b-1_x-struc_y_unstruct}
  %     \\
  %     \FEthreeSpacesP{1,b}{1,b}1
  %     &
  %     \FEthreeSpacesP{1,b}{1}1
  %     &
  %     \FEthreeSpacesP{1}{1,b}1
  % \end{tabular}
%   \end{center}
% \end{frame}


% \begin{frame}{\myframedAlert{\FEfourSpacesP11{1,b}1} 3D
%     finite elements}
%   ¿Cual de las simulaciones 3D meto?
% \end{frame}

\begin{frame}{\myframedAlert{\FEthreeSpacesP{2}11}
    stability}
  \begin{enumerate}\itemsep0.66em
  \item Methods and results: parallel to \FEthreeSpacesP{1,b}11
    (although more technical)
  \item We have also proved that \FEthreeSpacesP{2}11 is stable in
    most unstructured 2D meshes
  \end{enumerate}
\end{frame}

\begin{frame}{Macro-condition for \FEthreeSpacesP{2}{1}{1}}
  \begin{theorem}
    Let $M$ be a macroelement in $\MhOneVertex$.
    Then \textbf{Macro-condition} is verified for \colorUnderLine{\FEthreeSpacesP{2}{1}{1} is
      in $M$}
    $\Leftrightarrow$ $M$ is
    \colorUnderLine{\textbf{\yUnstructured}} except...

    \begin{quote}
      $\nv$ is even and the following algebraic condition
      holds \vspace{-0.6em} {\scriptsize
        \begin{equation*}
          \label{eq:ch1:instabilityMacroCondition}
          \sum_{i=1}^\nv (-1)^{i} \cot(\sigma_i)\left(\frac1{|T_i|}+ \frac1{|T_{i+1}|} \right) = 0,
        \end{equation*}
      }
      \vspace{-0.6em}
      {\scriptsize
        where $\sigma_i$: angle between positive horizontal
        semiaxis from $q_0$ and  common edge of $T_i$ and $T_{i+1}$
      }
    \end{quote}
  \end{theorem}


\textbf{Proof}: Similar to \FEthreeSpacesP{1,b}{1}{1}, but
more technical \hfill $\square$
\par\mbox{}
\end{frame}

% \begin{frame}
%   \begin{proof}
%     \begin{itemize}
%     \item Es m\'as complicada que \FEthreeSpacesP{1,b}11
%     \item Idea: Usar la de  mi charla para el grupo de Tom\'as
%       (repasarla)
%     \end{itemize}
%   \end{proof}
% \end{frame}

  \newcommand{\macroelemD}{
    \begin{tikzpicture}
      \def\nv{6}
      \def\ix{1.75}\def\iy{1.25}
      \def\vertices{
        \coordinate [label=-45:$q_0$] (q0) at (0,0);
        \coordinate [label=right: $q_1$] (q1) at (\ix*1,0);
        \coordinate [label=+45: $q_2$] (q2) at (\ix,\iy);
        \coordinate [label=+135: $q_3$] (q3) at  (0,\iy);
        \coordinate [label=+135: $q_4$] (q4) at   (-\ix,0);
        \coordinate [label=225: $q_5$] (q5) at (-\ix,-\iy);
        \coordinate [label=-45: $q_6$] (q6) at (0,-\iy);
      }
      \def\xPoints {
        \coordinate (x1) at ($ (q1)!.5!(q0) $);
        \coordinate (x2) at ($ (q2)!.5!(q0) $);
        \coordinate (x3) at ($ (q3)!.5!(q0) $);
        \coordinate (x4) at ($ (q4)!.5!(q0) $);
        \coordinate (x5) at ($ (q5)!.5!(q0) $);
        \coordinate (x6) at ($ (q6)!.5!(q0) $);
      }
      \def\showPoints{
        \foreach \i in {0,...,\nv}
        \fill [black] (q\i) circle (3pt);
        \foreach \i in {1,...,\nv} {
          \draw [fill, lightgray] (x\i) circle (2.25pt);
          \draw (x\i) circle (2.25pt);
        }
      }
      \def\showLines{
        % Draw borders
        \draw (q1)--(q2) -- (q3) -- (q4) -- (q5) -- (q6) -- cycle;
        % Draw interior edges
        \foreach \i in {1,...,\nv} {
          \draw (q\i) -- (q0);
        }
      }
      \def\showTriangles{
        \node at ($ (q6)!.5!(q1) $) [label=-45:${T_6}$] {};
        \node at ($ (q1)!.5!(q2) $) [label=0:${T_1}$] {};
        \node at ($ (q2)!.5!(q3) $) [label=90:${T_2}$] {};
        \node at ($ (q3)!.5!(q4) $) [label=135:${T_3}$] {};
        \node at ($ (q4)!.5!(q5) $) [label=180:${T_4}$] {};
        \node at ($ (q5)!.5!(q6) $) [label=-45:${T_5}$] {};
      }
      \vertices
      \xPoints
      \showPoints
      \showLines
      \showTriangles
    \end{tikzpicture}
  }
  \newcommand{\macroelemE}{
    \begin{tikzpicture}
      \def\nv{6}
      \def\ix{1.75}\def\iy{0.75}
      \def\vertices{
        \coordinate [label=-45:$q_0$] (q0) at (0,0);
        \coordinate [label=right: $q_1$] (q1) at (\ix,\iy);
        \coordinate [label=+45: $q_2$] (q2) at (0,2*\iy);
        \coordinate [label=+135: $q_3$] (q3) at  (-\ix,\iy);
        \coordinate [label=+135: $q_4$] (q4) at   (-\ix,-\iy);
        \coordinate [label=225: $q_5$] (q5) at (0,-2*\iy);
        \coordinate [label=-45: $q_6$] (q6) at (\ix,-\iy);
      }
      \def\xPoints {
        \coordinate (x1) at ($ (q1)!.5!(q0) $);
        \coordinate (x2) at ($ (q2)!.5!(q0) $);
        \coordinate (x3) at ($ (q3)!.5!(q0) $);
        \coordinate (x4) at ($ (q4)!.5!(q0) $);
        \coordinate (x5) at ($ (q5)!.5!(q0) $);
        \coordinate (x6) at ($ (q6)!.5!(q0) $);
      }
      \def\showPoints{
        \foreach \i in {0,...,\nv}
        \fill [black] (q\i) circle (3pt);
        \foreach \i in {1,...,\nv} {
          \draw [fill, lightgray] (x\i) circle (2.25pt);
          \draw (x\i) circle (2.25pt);
        }
      }
      \def\showLines{
        % Draw borders
        \draw (q1)--(q2) -- (q3) -- (q4) -- (q5) -- (q6) -- cycle;
        % Draw interior edges
        \foreach \i in {1,...,\nv} {
          \draw (q\i) -- (q0);
        }
      }
      \def\showTriangles{
        \node at ($ (q6)!.5!(q1) $) [label=-45:${T_1}$] {};
        \node at ($ (q1)!.5!(q2) $) [label=0:${T_2}$] {};
        \node at ($ (q2)!.5!(q3) $) [label=90:${T_3}$] {};
        \node at ($ (q3)!.5!(q4) $) [label=135:${T_4}$] {};
        \node at ($ (q4)!.5!(q5) $) [label=180:${T_5}$] {};
        \node at ($ (q5)!.5!(q6) $) [label=-45:${T_6}$] {};
      }
      \vertices
      \xPoints
      \showPoints
      \showLines
      \showTriangles
    \end{tikzpicture}
  }


% \begin{frame}{{\amarillo Not regular} \FEthreeSpacesP211 macroelements}
%   \begin{columns}
%     %- - - - - - - - - - - -
%     \column[t]{0.49\linewidth}
%     %- - - - - - - - - - - -
%     % \subfloat[Two vertices
%     % horizontally aligned with $q_0$.]{
%     \macroelemD
%     \begin{flushleft}
%       \quad\alert{(a)}\quad \myframed{\yStructured}
%     \end{flushleft}
%     % \subfloat[No vertex horizontally aligned, but $n$ is even
%     % and~(\ref{eq:ch1:instabilityMacroCondition}) holds.]{
%     %- - - - - - - - - - - -
%     \column[t]{0.49\linewidth}
%     %- - - - - - - - - - - -
%     \macroelemE
%     \begin{figure}
%       \alert{(b)}\quad \myframed{\yUnstructured}
%       \\[0.66em]
%       (``\textit{weak}'' $y$--structure)
%     \end{figure}
%   \end{columns}
% \end{frame}

\begin{frame}{\myframedAlert{Stokes \FEthreeSpacesP{2}11}
    lid-driven cavity test}
  \begin{center}
    \begin{tabular}{rr}
      \pgfimage[width=0.2\linewidth]{\imgdir/unstruct-mesh}
      &
      \pgfimage[width=0.2\linewidth]{\imgdir/struct-mesh}
      \\[-0.2em]
      \pgfimage[width=0.458\linewidth]{\imgdir/21-1-p-uns}
      &
      \pgfimage[width=0.45\linewidth]{\imgdir/21-1-p-str}
      \\
      \scriptsize Unstructured mesh pressure \quad & \scriptsize Structured mesh pressure\quad
    \end{tabular}
    \par\mbox{}
  \end{center}
\end{frame}

\begin{frame}{Stokes \FEthreeSpacesP{2}11 error orders}
  \small
  \begin{columns}
    % - - - - - - - - - - - - -
    \column{0.50\linewidth}
    % - - - - - - - - - - - - -
    Orders in energy norms:
    \begin{itemize}
    \item<2-> \myframed<2>{$\OO(h^{\mathbf{2}})$ for $\uu$ and $\vv$
        in $\|\cdot\|_{L^2}$}
          \tikz[na] \coordinate(uv-l2-error-L);
    \item<3-> \myframed<3>{$\OO(h^{\mathbf{1}})$ for $\uu$ and $\vv$
        in $\|\cdot\|_{H^1}$}
          \tikz[na] \coordinate(uv-h1-error-L);
        \item<4-> \myframed<4>{$\OO(h^{\mathbf{1}})$ for $\pp$ in $\|\cdot\|_{L^2}$}
          \tikz[na] \coordinate(p-l2-error-L);
      \par\medskip
    % \item Note: Order $\OO(h^3)$ not reached for $\uu\sim\P2$ due to
    %   influence of $\vv\sim\P1$
    \end{itemize}
    % - - - - - - - - - - - - -
    \column{0.49\linewidth}
    % - - - - - - - - - - - - -
   \begin{tikzpicture}
      \node [inner sep=0pt,above right]{
        \pgfimage[width=\linewidth]{\imgdir/21-1-errors-loglog}
      };
      \path (2,2.1) coordinate (u-error-img);
      \path (2,3) coordinate (v-h1z-error-img);
      \path (2,1.8) coordinate (v-l2-error-img);
    \end{tikzpicture}
  \end{columns}
  % - - - - - - - - - - - - - - - - - - - - - - - - - - - - - - - - -
  \vspace{-1em}
  \begin{overprint}
    \begin{center}
      \scriptsize
      \begin{tabular}{l@{\qquad}r@{\qquad}rrrrrr}
        \toprule
        $h_{\max}$ & & $2^{-3}$ & $2^{-4}$ & $2^{-5}$ & $2^{-6}$ & $2^{-7}$
        & $2^{-8}$
        \\ \otoprule
        \multirow{2}*{$\uu$}
        & $\|u-u_h\|_{L^2}$ &
        \rowalert{2} {2.185  }{ 2.291  }{ 2.008} {
          \tikz[na]  \node(u-l2-error-R) {2.029  };
        }{2.031  }{ 2.066}
        \\ \cmidrule{2-8}
        & $\|u-u_h\|_{H^1_0}$ &
        \rowalert{3} {1.758  }{ 1.774  }{ 0.986} {
          \tikz[na]  \node(u-h1-error-R) {1.212};
        }{1.067  }{ 1.237}
        \\ \midrule
        \multirow{2}*{$\vv$}
        & $\|v-v_h\|_{L^2}$ &
        \rowalert{2} {2.056  }{ 2.116  }{ 2.028} { \tikz[na]
          \node(v-l2-error-R) {1.980};  }{ 2.068  }{ 2.040}
        \\ \cmidrule{2-8}
        & $\|\dz v-\dz v_h\|_{L_2}$ &
        \rowalert{3} {0.977  }{ 1.044  }{ 1.017}{ \tikz[na]
          \node(v-h1z-error-R){0.990}; }{1.022  }{ 1.019}
        \\ \midrule
        $\pp$
        & $\|p-p_h\|_{L^2}$ &
        \rowalert{4} {1.070  }{ 2.101  }{ 0.798}{ \tikz[na]
          \node(p-l2-error-R){1.419};}{0.599  }{ 0.820}
        \\ \bottomrule
      \end{tabular}
    \end{center}
  \end{overprint}
  \begin{tikzpicture}[overlay]
    \path<2>[myarrow]
    (u-l2-error-R) edge [in=0,out=155] (uv-l2-error-L);
    % \path<2>[myarrow2]
    % (u-error-L) edge [out=0, in=170] (u-error-img);
    %
    \path<2>[myarrow]
    (v-l2-error-R) edge [in=0,out=155] (uv-l2-error-L);
    % \path<3>[myarrow2]
    % (v-h1z-error-L) edge [out=0, in=170] (v-h1z-error-img);
    % ------------------------------------------------------
    \path<3>[myarrow]
    (u-h1-error-R) edge [out=155, in=0] (uv-h1-error-L);
    %
    % \path<2>[myarrow2]
    % (u-error-L) edge [out=0, in=170] (u-error-img);
    \path<3>[myarrow]
    (v-h1z-error-R) edge [out=155, in=0] (uv-h1-error-L);
    % \path<3>[myarrow2]
    % (v-h1z-error-L) edge [out=0, in=170] (v-h1z-error-img);
    % ------------------------------------------------------
    \path<4>[myarrow]
    (p-l2-error-R) edge [out=155, in=0] (p-l2-error-L);
    % \path<4>[myarrow2]
    % (p-l2-error-L) edge [out=0, in=155] (v-l2-error-img);
  \end{tikzpicture}
\end{frame}


\begin{frame}{3D finite elements}

  Similar results are shown in \textbf{3D} when one delete \textbf{bubbles}...
  \begin{itemize}\itemsep0.7em
  \item \FEfourSpacesP{1,b}{1,b}{1}{1}
  \item \FEfourSpacesP{1,b}{1}{1}{1}
  \end{itemize}

  \medskip
  And also when one delete degrees of freedom in \textbf{3D Taylor-Hood} elements
  \begin{itemize}\itemsep0.7em
  \item  \FEfourSpacesP{2}{2}{1}{1}
  \item \FEfourSpacesP{2}{1}{1}{1}
  \end{itemize}
  \bigskip
  \textbf{Proof}: Similar 2D but quite technical in some cases \hfill $\square$
\end{frame}

\begin{frame}{3D tests (lid-driven cavity tests)}

  \textbf{Structured mesh}
  \begin{tabular}{ccc}
    \pgfimage[width=0.3\linewidth]{\imgdir/222-1-str}
    &
    \pgfimage[width=0.3\linewidth]{\imgdir/221-1-str}
    &
    \pgfimage[width=0.3\linewidth]{\imgdir/211-1-str}
  \end{tabular}
  \textbf{Unstructured mesh}
  \begin{tabular}{ccc}
    \pgfimage[width=0.3\linewidth]{\imgdir/222-1-uns}
    &
    \pgfimage[width=0.3\linewidth]{\imgdir/221-1-uns}
    &
    \pgfimage[width=0.3\linewidth]{\imgdir/211-1-uns}
    \\[-0.3em]
    \small\FEfourSpacesP2221 & \small\FEfourSpacesP2211 &
    \small\FEfourSpacesP2111
  \end{tabular}
\end{frame}



%%% Local Variables:
%%% coding: utf-8
%%% TeX-master: "numerical-oceanography"
%%% mode: latex
%%% ispell-local-dictionary: "english"
%%% End:
