\begin{frame}{Mathematical Modelling and Numerical Simulation}
  \begin{itemize}
  \item \alert{Mathematical Model}: representation of physical reality
    that is suitable for \textit{analysis} and \textit{calculation}
  \item \alert{Numerical Simulation} allows us to \textit{calculate the
    solution of a model} in a computer and therefore to simulate
    physical reality
  \end{itemize}
  \medskip
  In this work\dots
  \smallskip
  \begin{itemize}
  \item models defined by \textbf{PDEs} (Partial Differential Equations)
  \item we shall introduce models of \textit{increasing complexity}
  \item targets:
    \begin{itemize}
    \item to explore and \textit{analyze} models in large-scale
      \textbf{Oceanography}
    \item to develop \textit{numerical methods} for simulate
      ocean flow
    \end{itemize}
  \end{itemize}
\end{frame}

\begin{frame}{A First Example of Modelling}
  Modelling requires \alert{great effort} for an applied mathematician
  \begin{itemize}
  \item Thorough knowledge of \llaveizq{\textit{mathematics}\\\textit{scientific discipline}}
  \item To communicate with other scientists, to select the adequate
    model between different possibilities, to simplify models which
    are too complex\dots
  \item In later stage, also \textit{computing skills}
  \end{itemize}

  \bigskip
  We start describing a simple model: the \alert{Heat Equation}
\end{frame}

\SetEmptyBackground
\begin{frame}
  Let
  \begin{itemize}
  \item $\Omega\subset\Rset^N$, \quad ($N=1,2,3$), \ open spatial domain \tiragris{occupied by a
      homogeneous, isotropic, heat conductor material}
  \item $\xx=(x_1,\dots,x_N) \in \Omega$: \textit{space} variable, \quad $t\in (0,T)$: \textit{time} variable
  \item \textit{Temperature}: unknown function,
    \begin{align*}
    u:\Omega\times (0,T) \to \Rset, \\
      (\xx,t) \mapsto u(\xx,t).
    \end{align*}
    \vspace{-1em}
    \pause
  \item \textit{Heat source}: a given function, $f(\xx,t)$
    \vspace{0.66em}
  \item Using physical laws (see e.g.~\cite{allaire:2007}),
    \begin{itemize}
    \item conservation of energy (of heat),
    \item quantity of heat is proportional to temperature: $c u$, \ $c\in\Rset$,
    \item heat flux proportional to temperature gradient: $q=-k\grad u$, $k\in\Rset$,
      \tiragris{\tiny $c$, $k$: physical
        constants (dependent on material), called ``specific heat'' and ``thermal conductivity''}
    \end{itemize}
  \item \dots and some mathematical theory (Gauss theorem), we arrive to\dots
  \end{itemize}
\end{frame}

\SetDefaultBackground
\begin{frame}{The Heat Equation}
  \begin{itemize}
  \item ... we arrive to the following \textbf{PDE} in $\Omega$:
    \begin{BlockNoTitle}
      \begin{equation*}
        c\, \framedmath<2>{\dt u} - k\, \framedmath<3>{\Delta u} = f
        \qquad {\color{PHDgray} \text{in} \quad \Omega\times (0,T),}
      \end{equation*}
    \end{BlockNoTitle}
    where:
    \begin{itemize}
    \item $\framedmath<2>{\dt u} = \frac{\partial u}{\partial t}$
      \quad (time partial derivative)
    \item
      $\framedmath<3>{\Delta u} = \sum_{i=1}^N \frac{\partial^2
        u}{\partial x_i^2}$ \quad (Laplacian operator)
    \end{itemize}
    \bigskip
    \item<4-> We must add an \textbf{initial condition}
      {\color{PHDgray}(state of the model at time $t=0$)}:
      \begin{BlockNoTitle}
        \begin{equation*}
          u(\xx,0) = u_0(\xx), {\color{PHDgray} \qquad \forall \xx\in\Omega}
        \end{equation*}
      \end{BlockNoTitle}

    \item<4->
      And also \textbf{boundary conditions}
      {\color{PHDgray}(evolution of the model on the boundary,
        $\xx\in\partial\Omega$)}\dots
  \end{itemize}
\end{frame}

\begin{frame}{Boundary Conditions (on $\partial\Omega$)}
  Different possibilities {\color{PHDgray} (depending on physical context)}:
  \begin{itemize}
  \item \alert{Dirichlet} boundary condition
    \begin{BlockNoTitle}
      \begin{equation*}
        u(\xx,t)=0, \quad \forall \xx\in\partial\Omega, \quad \forall t\in(0,T)
      \end{equation*}
    \end{BlockNoTitle}
  \item \alert{Neumman} boundary condition
  \item \alert{Robin} boundary condition
  \end{itemize}
\end{frame}

%%% Local Variables:
%%% coding: utf-8
%%% TeX-master: "numerical-oceanography"
%%% mode: latex
%%% ispell-local-dictionary: "english"
%%% End:
