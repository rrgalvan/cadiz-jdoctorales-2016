\begin{frame}{Mathematical Modelling and Numerical Simulation}
  \begin{itemize}
  \item \alert{Mathematical Model}: representation of physical reality
    that is suitable for \textit{analysis} and \textit{calculation}
  \item \alert{Numerical Simulation} allows us to \textit{calculate the
    solution of a model} in a computer and therefore to simulate
    physical reality
  \end{itemize}
  \medskip
  In this work\dots
  \smallskip
  \begin{itemize}
  \item models defined by \textbf{PDEs} (Partial Differential Equations)
  \item we shall introduce models of \textit{increasing complexity}
  \item targets:
    \begin{itemize}
    \item to explore and \textit{analyze} models in large-scale
      \textbf{Oceanography}
    \item to develop \textit{numerical methods} for simulate
      ocean flow
    \end{itemize}
  \end{itemize}
\end{frame}

\begin{frame}{A First Example of Modelling}
  Modelling requires \alert{great effort} for an applied mathematician
  \begin{itemize}
  \item Thorough knowledge of \llaveizq{\textit{mathematics}\\\textit{scientific discipline}}
  \item To communicate with other scientists, to select the adequate
    model between different possibilities, to simplify models which
    are too complex\dots
  \item In later stage, also \textit{computing skills}
  \end{itemize}

  \bigskip
  We start describing a simple model: the \alert{Heat Equation}
\end{frame}

\SetEmptyBackground
\begin{frame}
  Let
  \begin{itemize}
  \item $\Omega\subset\Rset^N$, \quad ($N=1,2,3$), \ open spatial domain \tiragris{occupied by a
      homogeneous, isotropic, heat conductor material}
  \item $\xx=(x_1,\dots,x_N) \in \Omega$: \textit{space} variable, \quad $t\in (0,T)$: \textit{time} variable
  \item \textit{Temperature}: unknown function,
    \begin{align*}
    u:\Omega\times (0,T) \to \Rset, \\
      (\xx,t) \mapsto u(\xx,t).
    \end{align*}
    \vspace{-1em}
    \pause
  \item \textit{Heat source}: a given function, $f(\xx,t)$
    \vspace{0.66em}
  \item Using physical laws (see e.g.~\cite{allaire:2007}),
    \begin{itemize}
    \item conservation of energy (of heat),
    \item quantity of heat is proportional to temperature: $c u$, \ $c\in\Rset$,
    \item heat flux proportional to temperature gradient: $q=-k\grad u$, $k\in\Rset$,
      \tiragris{\tiny $c$, $k$: physical
        constants (dependent on material), called ``specific heat'' and ``thermal conductivity''}
    \end{itemize}
  \item \dots and some mathematical theory (Gauss theorem), we arrive to\dots
  \end{itemize}
\end{frame}

\SetDefaultBackground
\begin{frame}{The Heat Equation}
  \begin{itemize}
  \item ... we arrive to the following \textbf{PDE} in $\Omega$:
    \begin{BlockNoTitle}
      \begin{equation*}
        c\, \framedmath<2>{\dt u} - k\, \framedmath<3>{\Delta u} = f
        \qquad {\color{PHDgray} \text{in} \quad \Omega\times (0,T),}
      \end{equation*}
    \end{BlockNoTitle}
    where:
    \begin{itemize}
    \item $\framedmath<2>{\dt u} = \frac{\partial u}{\partial t}$
      \quad (time partial derivative)
    \item
      $\framedmath<3>{\Delta u} = \sum_{i=1}^N \frac{\partial^2
        u}{\partial x_i^2}$ \quad (Laplacian operator)
    \end{itemize}
    \bigskip
    \item<4-> We must add an \textbf{initial condition}
      {\color{PHDgray}(state of the model at time $t=0$)}:
      \begin{BlockNoTitle}
        \begin{equation*}
          u(\xx,0) = u_0(\xx), {\color{PHDgray} \qquad \forall \xx\in\Omega}
        \end{equation*}
      \end{BlockNoTitle}

    \item<4->
      And also \textbf{boundary conditions}
      {\color{PHDgray}(evolution of the model on the boundary,
        $\xx\in\partial\Omega$)}\dots
  \end{itemize}
\end{frame}

\begin{frame}{Boundary Conditions {\small (to select depending on physical context)}}
  % \vspace{-1.5em}
  % Different possibilities to select {\color{PHDgray} (depending on physical context)}:
  % \medskip
  \begin{enumerate}
  \item \alert{Dirichlet} boundary condition:
    \begin{BlockNoTitle}
      \begin{equation*}
        u(\xx,t)=0, \quad \forall \xx\in\partial\Omega, \quad \forall t\in(0,T)
      \end{equation*}
    \end{BlockNoTitle}
  \item \alert{Neumman} boundary condition:
    \begin{BlockNoTitle}
      \begin{equation*}
        \partial_{\nn} u(\xx,t) := \grad u(\xx,t)\cdot\nn = 0,
      \end{equation*}
    \end{BlockNoTitle}
    {\small\hfill
     where $\grad u =$ gradient vector and $\nn=$ unit outward vector normal to $\Omega$.}
    \smallskip
    \tiragris{\scriptsize{}
        \textbf{Interpretation}: No heat flux across $\partial\Omega$ (domain isolated from the exterior)}
  \item \alert{Robin} (or Fourier) boundary condition
    \begin{BlockNoTitle}
      \begin{equation*}
        \partial_{\nn} u(\xx,t) + \alpha u(\xx,t) = 0,
      \end{equation*}
    \end{BlockNoTitle}
    {\small\hfill where $\alpha>0$ is a given constant}
    \tiragris{\scriptsize
      \begin{tabular}{r}
      \textbf{Interpretation}: Heat flux trough $\partial\Omega$ is proportional to
        the difference \\ of temperature between exterior and interior
      \end{tabular}
    }
  \end{enumerate}
\end{frame}

\begin{frame}{The Heat Equation}
  We can summarize in the following problem: find $u(\xx,t)$ such that
  \begin{BlockNoTitle}%
    \begin{tabular}[t]{l|>{$}l<{$}l}
       \rotatebox[origin=c]{40}{\small \heatProblem}
      &
        \begin{aligned}%
          \dt u - \nu\, \Delta u &= f
          \qquad \text{in} \ \Omega\times (0,T),
          \\\noalign{\smallskip}
          u|_{t=0} &= u_0
          \qquad \text{in}\ \Omega,
          \\\noalign{\smallskip}
          u&=0
          \qquad \text{on}\ \partial\Omega \times (0,T),
        \end{aligned}
    \end{tabular}
  \end{BlockNoTitle}
  where $\nu=k/c>0$ (viscosity coefficient) and we fixed Dirichlet B.C.
  \pause
  \small
  \begin{remark}
    This problem has a universal character and is applied to model \structure{many phenomena}:
    \begin{itemize}
    \item Heat propagation
    \item Diffusion of a density or concentration (of biological
      species, of pollution,\dots)
    \item In finance, value of an option to buy a stock
    \item \dots
    \end{itemize}
  \end{remark}
\end{frame}

\begin{frame}{The Poisson Problem}
  For certain choices of the source term, $f(x)$, Heat Equation reaches a
  \alert{\textbf{steady}} or (\textbf{stationary}) state, that is
  $$
  u(\xx,t) \longrightarrow u_\infty(\xx) \quad\mbox{as}\quad t\to\infty.
  $$
  Often it is interesting to calculate the steady state directly, solving:
  \begin{BlockNoTitle}%
    \begin{tabular}[t]{l|>{$}l<{$}l}
       \rotatebox[origin=c]{30}{\small \poissonProblem}
      &
        \begin{aligned}%
          \Delta u &= f
          \qquad \text{in} \ \Omega,
          \\\noalign{\smallskip}
          u&=0
          \qquad \text{on}\ \partial\Omega.
        \end{aligned}
    \end{tabular}
  \end{BlockNoTitle}
  \medskip
  \pause
  Now we are about to use this simpler problem as example for:
  \smallskip
  \begin{enumerate}
  \item \emph{Analysis} of the model (including existence and uniqueness of solution) and
  \item \emph{Numerical simulation} (calculating the solution).
\end{enumerate}
\end{frame}

\begin{frame}{Variational Formulation of \poissonProblem}
  \small
  \begin{proposition}
  Let
  $$
  X=\{ \Phi\in C^1(\overline\Omega) \st \Phi=0 \quad \mbox{on } \partial\Omega\}.
  $$
  A function $u\in C^2(\Omega)$ is a solution of \poissonProblem $\mathbf\Longleftrightarrow$ $u\in X$ and satisfies:
  \begin{equation}
    \tag{PV}
    \int_\Omega \grad u(\xx) \grad v(\xx) d\xx = \int_\Omega f(\xx) v(\xx) d\xx \quad \forall v\in X.
    \label{eq:poisson.variational}
  \end{equation}
  \end{proposition}
  \begin{itemize}
  \item Proof: follows from Green's formula (integration by parts).
  \item Problem (\ref{eq:poisson.variational}) is called
    \alert{\textbf{Variational Formulation}} of \poissonProblem.
  \item $v$ is called \textbf{test function}.
  \item (\ref{eq:poisson.variational}) only requires
    $u\in C^1(\Omega)$. In this sense it is \textbf{weaker} than ``classical
    formulation'' of \poissonProblem, which requires $u\in C^2(\Omega)$.
  \item $\Rightarrow$ we suspect that it is \textbf{easier to solve}
    (\ref{eq:poisson.variational}) than ``classical formulation''.
  \end{itemize}
  \begin{center}
    \textbf{Task:} to develop a theory for well-posedness
    of~(\ref{eq:poisson.variational}).
  \end{center}
\end{frame}

\begin{frame}{Abstract framework}
  \begin{itemize}
  \item Reformulation of (\ref{eq:poisson.variational}): find
    $u\in X$ such that
    \begin{BlockNoTitle}
      \begin{equation}
        \tag{PA}
        a(u,v) = L(v) \quad \forall v\in
        X,\label{eq:abstract.variational}
      \end{equation}
    \end{BlockNoTitle}
    were $a(\cdot,\cdot)$ and $L(\cdot)$ are bilinear and linear forms,
    respectively:
    \begin{equation*}
      a(u,v) = \int_\Omega \grad u \grad v, \qquad
      L(v) = \int_\Omega f  v.
    \end{equation*}

  \item There \textbf{exists a powerful theory}
    {\color{PHDgray}(e. g. Lax-Milgram)} for analysing abstract variational
    formulations like~(\ref{eq:abstract.variational}).

  \item However, this theory \textbf{only works for Hilbert spaces}. And
    $$
    X=\{ \Phi\in C^1(\overline\Omega) \st \Phi=0 \quad \mbox{on } \partial\Omega\}
    $$
    is \textbf{not a Hilbert space} (equipped with  the "natural" scalar product for this problem).
  \item We therefore must \textbf{replace $X$ with a Hilbert space}: $V=H_0^1(\Omega)$.
  \end{itemize}

\end{frame}

\begin{frame}{Lax-Milgram Theory}
  Let us consider the abstract variational problem: find $u\in V$ such that
    \begin{BlockNoTitle}
      \begin{equation}
        \tag{PA}
        a(u,v) = L(v) \quad \forall v\in V,
        \label{eq:abstract.variational.hilbert}
      \end{equation}
    \end{BlockNoTitle}
    \begin{enumerate}
    \item $L: V \to \Rset$ is linear and continuous
    \end{enumerate}
    \begin{theorem}[Lax-Milgram]
      Let $V$ be a real Hilbert space
      Then~(\ref{eq:abstract.variational.hilbert})
      has a unique solution.
    \end{theorem}
\end{frame}

\begin{frame}{Sobolev spaces}
  \small
  {\color{PHDgray} Brief summary of results from \textbf{functional analysis}:}
  \begin{itemize}
  \setlength\itemsep{0.5em}
  \item<1->
    $L^p(\Omega)=\{ f: \Omega \to \Rset \quad \text{measurable} \st
    \alert{\int_\Omega |f(\xx)|^p < +\infty} \} \quad p \in [1,+\infty)$
    \begin{itemize}
      \setlength\itemsep{0.3em}
      \scriptsize
    \item $L^p(\Omega)$ is a Banach space for
      $\norm[L^p(\Omega)]{f}=\left(\int_\Omega|f(\xx)|^p\right)^{1/p}$
    \item $L^2(\Omega)$ is a Hilbert space for $(f,g)_{L^2(\Omega)}=\int_\Omega f(\xx) g(\xx) d\xx$
    \end{itemize}

  \item<2->
    $ H^m(\Omega) = \{ f\in L^2(\Omega) \st \text{\alert{weak derivatives} of
      order $|\alpha| \le m$ are \alert{in} } \alert{L^2(\Omega)} \} $
    \begin{itemize}
      \setlength\itemsep{0.3em}
      \scriptsize
    \item Hilbert space for
      $(f,g)_{H^m(\Omega)}=\sum_{|\alpha|\le m} \int_\Omega D^\alpha f
      \; D^\alpha g \,d\xx$.
      \par {\color{PHDgray} which induces the norm:} \quad
      $\norm[H^m(\Omega)]{f}=\left(\sum_{|\alpha|\le m} \int_\Omega|D^\alpha f(\xx)|^2\right)^{1/2}$
      \item $C^\infty(\overline\Omega) \subset H^m(\Omega)$, dense inclusion.
    \end{itemize}

  \item<3-> \alert{$H_0^1(\Omega) = \{ v \in H^1(\Omega) \st v|_{\partial\Omega}=0 \}$}
     \hfill{\color{PHDgray} (sense of $v|_{\partial\Omega}=0$ ? Trace theory)}
    \begin{itemize}
      \setlength\itemsep{0.3em}
      \scriptsize
    \item Hilbert space for
      $(f,g)_{H_0^1(\Omega)}=\int_\Omega \grad f \; \grad g \, d\xx $
      \par {\color{PHDgray} which induces the norm:} \quad
      $\norm[H_0^1(\Omega)]{f}=\left(\int_\Omega|\grad f|^2\, d\xx\right)^{1/2}$
    \item $D(\Omega) =\{ \small \Phi\in C^\infty(\Omega) \st \text{ soporte compacto} \}
      \subset H_0^1(\Omega)$, dense.
       \small
     $$\Rightarrow
      D(\Omega) \subset \alert{X=\{ \Phi\in C^1(\overline\Omega) \st \Phi=0 \quad \mbox{on
      } \partial\Omega\} \subset H_0^1(\Omega)} \text{, dense}.$$
    \end{itemize}
  \end{itemize}
\end{frame}

%%% Local Variables:
%%% coding: utf-8
%%% TeX-master: "numerical-oceanography"
%%% mode: latex
%%% ispell-local-dictionary: "english"
%%% End:
